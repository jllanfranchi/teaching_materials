%  use 
%    xdvi -paper usr formulae &
%  and 
%    dvips -t landscape formulae
%  to preview and make PostScript
%,

\documentclass[letterpaper,landscape,10pt]{article}
\usepackage{multicol}
\usepackage{calc}
\usepackage{ifthen}
\usepackage[landscape]{geometry}
\usepackage{amsmath}
\usepackage{amssymb}
\usepackage{titlesec}
\usepackage{units}
\usepackage{caption}
\usepackage{fancyhdr}
\usepackage{tabularx}
%\usepackage{lastpage}
%\usepackage{pxfonts}
\usepackage{helvet}
\usepackage{mathptmx}
%\usepackage{times}
%\usepackage{type1cm}
%\usepackage[T1]{fontenc}
%\usepackage[charter]{mathdesign}
%\usepackage{fouriernc}

\usepackage{hhline}
\usepackage{colortbl}
\usepackage{mathrsfs}
\usepackage{bm}
\usepackage{color}
\usepackage{ifpdf}
\usepackage[protrusion=true,expansion=true]{microtype}
\usepackage{setspace}
\singlespacing

\input{header.tex}

\geometry{top=.85in,left=.32in,right=.32in,bottom=.30in}
\title{Math \& Physics Equation Sheet}
\author{Justin Lanfranchi}
\date{\today}
\renewcommand{\baselinestretch}{.5}

\ifpdf
	\usepackage[pdftex]{graphicx}
	\usepackage
	[colorlinks=true,
	linkcolor=webgreen, %defined below
	filecolor=webbrown, %defined below
	citecolor=webgreen, %defined below
	%------------- Doc Info ---------------------------------
	pdftitle={Math and Physics Equation Sheet},
	pdfauthor={Justin Lanfranchi},
	pdfsubject={},
	pdfkeywords={},
	%------------ Doc View ----------------------------------
	bookmarksopen=true,
	pdfpagemode=UseOutlines]{hyperref}
	\definecolor{webgreen}{rgb}{0,0,0}
	\definecolor{webbrown}{rgb}{0,0,0}
\else
	\usepackage{graphicx}
\fi

%\fancyhf{}
\setlength{\headheight}{8pt}
\setlength{\headsep}{5pt}
\renewcommand{\headrulewidth}{1.0pt}
\renewcommand{\footrulewidth}{0pt}
\lhead{}
%\chead{\fontsize{8}{1}{}\selectfont\textsc{Math \& Physics Equation Sheet}  \hfill \today \hfill Justin Lanfranchi{\hfill}Page \thepage\ of \pageref{LastPage}{}}
\chead{\fontsize{8}{1}{}\selectfont\textsc{Math \& Physics Equation Sheet}  \hfill \today \hfill Justin Lanfranchi{\hfill}Page \thepage}
\rhead{}
\lfoot{}
\cfoot{}
\rfoot{}
%\maketitle

\pagestyle{fancyplain}
\newif\iftechexplorer\techexplorerfalse
%\markright{ \hrulefill\ equation sheet, Page }




\makeatletter
\renewcommand*\env@matrix[1][*\c@MaxMatrixCols c]{%
	\hskip -\arraycolsep
	\let\@ifnextchar\new@ifnextchar
	\array{#1}}
\makeatother


% Alter some LaTeX defaults for better treatment of figures:
% See p.105 of "TeX Unbound" for suggested values.
% See pp. 199-200 of Lamport's "LaTeX" book for details.
%   General parameters, for ALL pages:
\renewcommand{\topfraction}{1.0}    % max fraction of floats at top
\renewcommand{\bottomfraction}{1.0} % max fraction of floats at bottom
%   Parameters for TEXT pages (not float pages):
\setcounter{topnumber}{8}
\setcounter{bottomnumber}{8}
\setcounter{totalnumber}{16}     % 2 may work better
\setcounter{dbltopnumber}{16}    % for 2-column pages
\renewcommand{\dbltopfraction}{0.9} % fit big float above 2-col. text
\renewcommand{\textfraction}{0.07}  % allow minimal text w. figs
%   Parameters for FLOAT pages (not text pages):
\renewcommand{\floatpagefraction}{0.7}  % require fuller float pages
% N.B.: floatpagefraction MUST be less than topfraction !!
\renewcommand{\dblfloatpagefraction}{0.7}   % require fuller float pages

% remember to use [htp] or [htpb] for placement


%\titleformat{\section}[display]
%	{\scshape\normalsize\filcenter}
%	{\thesection}
%	{1pt}
%	{\titlerule \vspace{2pt} \small}
%	[\vspace{1pt} \titlerule]

\titleformat{\section}[display]
	{\scshape\Huge\filcenter}
	{\thesection}
	{1pt}
	{\vspace{3pt} \small}
	[\vspace{3pt}]
 
\titleformat{\subsection}{\small\sffamily}{\thesubsection}{1em}{}{}{}
\titleformat{\subsubsection}{\scriptsize\slshape}{\thesubsubsection}{1em}{}{}{}

\titlespacing*{\section}      {-8pt}{2.5pt}{2.5pt}
\titlespacing*{\subsection}   {-6pt}{5pt}{2pt}
\titlespacing*{\subsubsection}{-4pt}{5pt}{2pt}

%\titlespacing*{\section}      {-0pt}{2.5pt}{2.5pt}
%\titlespacing*{\subsection}   {-0pt}{5pt}{2pt}
%\titlespacing*{\subsubsection}{-0pt}{5pt}{2pt}


\newenvironment{mydescription}
{\begin{description}
	\setlength{\itemsep}{0pt}
	\setlength{\parskip}{0pt}
	\setlength{\parsep}{-1pt}}
{\end{description}}

%WAS:
%\newenvironment{myitemize}
%{\begin{itemize}
%	\setlength{\itemsep}{-1pt}
%	\setlength{\parskip}{0pt}
%	\setlength{\parsep}{0pt}}
%{\end{itemize}}

%NOW:
\newenvironment{myitemize}
{\begin{itemize}
	\setlength{\itemsep}{0pt}
	\setlength{\parskip}{0pt}
	\setlength{\parsep}{0pt}}
{\end{itemize}}

\newenvironment{titemize}
{\begin{list}{$\cdot$}{\leftmargin=1em}
	\setlength{\itemsep}{0pt}
	\setlength{\parskip}{0pt}
	\setlength{\parsep}{0pt}}
{\end{list}}

\newenvironment{litemize}
{\begin{list}{$\cdot$}{\leftmargin=6em\labelwidth=6em}
	\setlength{\itemsep}{0pt}
	\setlength{\parskip}{0pt}
	\setlength{\parsep}{0pt}}
{\end{list}}

\begin{document}{
\raggedright

%\fontsize{5}{1}\usefont{OT1}{cmr}{m}{n}\selectfont
%\fontsize{7}{1}\selectfont %\usefont{times}\selectfont

\fontsize{6}{1}\selectfont %\usefont{T1}{times}\selectfont
%\fontsize{10}{1}\selectfont %\usefont{T1}{times}\selectfont

\begin{multicols}{4}
%\begin{multicols}{3}

%\twocolumn

%\setlength{\premulticols}{18pt}
%\setlength{\postmulticols}{18pt}
%\setlength{\multicolsep}{18pt}
%\setlength{\columnsep}{18pt}
\iftechexplorer
  \maketitle
\fi


\section*{Quantum Mechanics}
  \subsubsection*{de Broglie}
	\hspace{5pt}for ALL things, light \& matter: \\
	\hspace{15pt}$\lambda=h/p$ \\
	\hspace{15pt}$\nu=E/h$
  \subsubsection*{Schr\"odinger Equation}
	$$ \hat{H} \ket{\Psi} = i\hbar\pd{}{t}\ket{\Psi} $$
  	%$$ i\hbar\frac{\partial\Psi}{\partial t} = -\frac{\hbar^2}{2m}\frac{\partial^2\Psi}{\partial x^2} + V\Psi$$
	$$ \hat{H} \ket{\psi(\xi)\phi(t)} = E_n \ket{\psi(\xi)} e^{iE_nt/\hbar} $$
	\hspace{5pt}this is non-relativistic\\
	\hspace{5pt}$\Psi$ can be complex-valued\\
	\hspace{5pt}boundary conditions lead to energy quantization\\
	\hspace{5pt}not derived (initially) but fit to reality\\
  \subsubsection*{Time Independent Schr\"odinger Equation}
  	$$ -\frac{\hbar^2}{2m}\frac{\partial^2\psi}{\partial x^2} + V\psi = E\psi$$
  %\subsubsection*{Quantum Things}
  %	\begin{titemize}
  %	  \item Mass, charge, etc.
  %    \item Energy (cf. blackbody radiation, photoelectric effect,
  %  	  photons \& other fields)
  %    \item Interference -- 2 slit experiment
  %	  \item Tunneling (radioactive decay, electronic devices)
  %	  \item Zero-point motion, i.e., electron never at 0 energy 
  %	  \item Diffraction in matter
  %	\end{titemize}
  \subsubsection*{Operators}
  	\hspace{5pt}\textbf{position}: $\langle x \rangle$: $\hat{x} = x$ \\
	\hspace{5pt}\textbf{function of position}: $\langle f(\bm{r})\rangle$: $\hat{f}=f(\bm{r})$ \\
	\hspace{5pt}\textbf{velocity}: $\langle v\rangle=\frac{d\langle x \rangle}{dt}$: $\hat{\bm{v}} = \frac{\hbar}{im}\bm\nabla$ \\
	\hspace{10pt}Note that this is velocity of expectation, but gives velocity in QM\\
	\hspace{5pt}\textbf{momentum}: $m\frac{d\langle x \rangle}{dt}$: $\hat{\bm{p}}=\frac{\hbar}{i}\bm\nabla$ \\
	\hspace{5pt}\textbf{energy}: $\hat H = -\frac{\hbar^2}{2m}\bm\nabla^2+V$ \\
	\hspace{5pt}\textbf{exchange}: $\hat P_{12} \equiv \hat P_{12} \ket{\Psi(\xi_1,\xi_2)} = \ket{\Psi(\xi_2,\xi_1)}$; eig. val's are $\pm 1\forall\ket{\Psi}$ known to man; $\hat P_{12}\left( \hat P_{12} \ket{\Psi(\xi_1,\xi_2)} \right) = \ket{\Psi(\xi_1,\xi_2)}$ \\
	\hspace{5pt}\textbf{parity}: $\hat \Pi \equiv \hat \Pi \ket{\Psi(\vec{r})} = \ket{\Psi(-\vec{r})}$; eig. val's are $\pm 1$ if they $\exists$ since $\hat\Pi \left(\hat\Pi \ket{\Psi(\vec{r})}\right) = \ket{\Psi(\vec{r})}$ \\
	\hspace{10pt} $\hat\Pi(Y_{lm}) = (-1)^l Y_{lm}$
	\hspace{5pt}\textbf{raising/lowering SHO}:\\
	\hspace{5pt}\textbf{ang. momentum}: $\bm{J} = $ \\

  \subsection*{Many-particle systems}
  	\subsubsection*{General solution}
	\subsubsection*{Special case: no external forces}
	\subsubsection*{Identical particles}
	\subsubsection*{Slater determinant}
	\subsubsection*{Pauli exclusion principle}
	\subsubsection*{Fermi energy -- bosonic system, non-interacting}
	\subsubsection*{Fermi energy -- fermionic system, non-interacting}
  \subsection*{Angular momentum and spin}
  	\subsubsection*{Spin paramagnetic resonance}
		\hspace{5pt} Rabi frequency \\
		\hspace{5pt} Rotating-wave approximation \\
  	\subsubsection*{Angular momentum / spin operators}
		\hspace{5pt} Operators \\
		\hspace{5pt} Addition \\
		\hspace{5pt} Clebsch-Gordon coefficients \\
		\hspace{5pt} Uncoupled representation \\
		\hspace{5pt} Coupled representation \\
		\hspace{5pt} Imposing conservation of angular momentum, parity, idential particle (anti-) symmetry on reaction \\
  \subsection*{Hydrogen-like atoms}
	\hspace{5pt} $H^0 = -\frac{\hbar^2}{2m_e}\nabla^2-\frac{kZe^2}{r}$ \\
	\hspace{5pt} $H^{rel}=-\frac{p^4}{8m_e^3c^2}$ \\
	\hspace{5pt} $H^{s-o}=\frac{kZe^2}{2m_e^2c^2}\frac{1}{r^3}\vec{s}\cdot\vec{L}$ \\
	\hspace{5pt} $H^{hf}=\frac{Ze^2}{4\pi\varepsilon_0}\frac{g_N}{4M_Nm_ec^2}\left( \frac{3\vec{r}(\vec{r}\cdot\vec{I})}{r^5} + \frac{8\pi}{3}\vec{I}\delta(\vec{r}) \right)\cdot(\vec{L}+2\vec{s})$ \\
	\hspace{5pt} 
	\hspace{5pt} 



\end{multicols}
}\end{document}
