%  use 
%    xdvi -paper usr formulae &
%  and 
%    dvips -t landscape formulae
%  to preview and make PostScript
%,

\documentclass[letterpaper,landscape,10pt]{article}
\usepackage{multicol}
\usepackage{calc}
\usepackage{ifthen}
\usepackage[landscape]{geometry}
\usepackage{amsmath}
\usepackage{amssymb}
\usepackage{titlesec}

%\usepackage{pxfonts}

%\usepackage{helvet}
\usepackage{mathptmx}
\usepackage{times}

%\usepackage{type1cm}
%\usepackage[T1]{fontenc}
%\usepackage{millennial}
%\usepackage[charter]{mathdesign}
%\usepackage{fouriernc}
\usepackage{hhline}
\usepackage{colortbl}
\usepackage{mathrsfs}
\usepackage{bm}

\titleformat{\section}[display]
	{\scshape\normalsize\filcenter}
	{\thesection}
	{1pt}
	{\titlerule \vspace{2pt} \small}
	[\vspace{1pt} \titlerule]
 
\titleformat{\subsection}{\small\sffamily}{\thesubsection}{1em}{}{}{}
\titleformat{\subsubsection}{\scriptsize\slshape}{\thesubsubsection}{1em}{}{}{}

\titlespacing*{\section}      {-10pt}{2.5pt}{2.5pt}
\titlespacing*{\subsection}   {-8pt}{5pt}{2pt}
\titlespacing*{\subsubsection}{-4pt}{5pt}{2pt}

\geometry{top=.20in,left=.32in,right=.32in,bottom=.30in}
\title{Equation Sheet}
\author{Justin Lanfranchi}
\date{2009.10.31}
\renewcommand{\baselinestretch}{.5}
\newif\iftechexplorer\techexplorerfalse
\markright{ \hrulefill\ equation sheet, Page }


\newenvironment{mydescription}
{\begin{description}
	\setlength{\itemsep}{0pt}
	\setlength{\parskip}{0pt}
	\setlength{\parsep}{-1pt}}
{\end{description}}

\newenvironment{myitemize}
{\begin{itemize}
	\setlength{\itemsep}{-1pt}
	\setlength{\parskip}{0pt}
	\setlength{\parsep}{0pt}}
{\end{itemize}}


\begin{document}{
\raggedright

%\fontsize{5}{1}\usefont{OT1}{cmr}{m}{n}\selectfont
%\fontsize{7}{1}\selectfont %\usefont{times}\selectfont
%\pagestyle{headings}
\fontsize{7}{1}\selectfont %\usefont{T1}{times}\selectfont
\begin{multicols}{3}
%\setlength{\premulticols}{18pt}
%\setlength{\postmulticols}{18pt}
%\setlength{\multicolsep}{18pt}
%\setlength{\columnsep}{18pt}
\iftechexplorer
  \maketitle
\fi

\section*{Mathematics}
%
%	\subsection*{Quadratic equation}
%		Solution to $ax^2+bx+c=0$: $x = \frac{-b \pm \sqrt{b^2-4ac}}{2a}$

%	\subsection*{Euler Identities}
%		{\centering
%		$Ae^{i\phi} = A(\cos\phi + i\sin\phi)$ \\
%		$e^{z} = e^{x+iy} = e^{x}(\cos y + i \sin y)$  \\
%		$\sin\phi = \frac{e^{i\phi}-e^{-i\phi}}{2i}$ \\
%		$\cos\phi = \frac{e^{i\phi}+e^{-i\phi}}{2}$ \\
%		}
%		%\begin{eqnarray*}
%		%	Ae^{i\phi} = A(\cos\phi + i\sin\phi)\\
%		%	e^{z} = e^{x+iy} = e^{x}(\cos y + i \sin y)\\
%		%	\sin\phi = \frac{e^{i\phi}-e^{-i\phi}}{2i}\\
%		%	\cos\phi = \frac{e^{i\phi}+e^{-i\phi}}{2}
%		%\end{eqnarray*}

%	\subsection*{Trig}
%	\textbf{Identities:}
%	\vspace{-5pt}
%	\begin{center}
%		$\displaystyle\sin (A\pm B) =  \sin A\cos B\pm \cos A\sin B$ \\
%		$\displaystyle\cos (A\pm B) =  \cos A\cos B\mp \sin A\sin B$ \\
%		\vspace{4pt}
%		$\displaystyle\sin A\sin B =
%			\frac{1}{2}\left[\cos(A-B)-\cos(A+B)\right]$\\
%		$\displaystyle\cos A\cos B =
%			\frac{1}{2}\left[\cos(A-B)+\cos(A+B)\right]$\\
%		$\displaystyle\sin A\cos B =
%			\frac{1}{2}\left[\sin(A+B)+\sin(A-B)\right]$\\
%		$\displaystyle\cos A\sin B =
%			\frac{1}{2}\left[\sin(A+B)-\sin(A-B)\right]$\\
%		\vspace{4pt}
%		$\displaystyle\sin^{2}A =  \frac{1-\cos 2A}{2}$ \\
%		$\displaystyle\cos^{2}A =  \frac{1+\cos 2A}{2}$ \\
%		\vspace{4pt}
%		$\displaystyle\sin A + \sin B =  2\sin\left(\frac{A+B}{2}\right)
%			\cos\left(\frac{A-B}{2}\right)$ \\
%		$\displaystyle\sin A - \sin B =  2\cos\left(\frac{A+B}{2}\right)
%			\sin\left(\frac{A-B}{2}\right)$ \\
%		$\displaystyle\cos A + \cos B =  2\cos\left(\frac{A+B}{2}\right)
%			\cos\left(\frac{A-B}{2}\right)$ \\
%		$\displaystyle\cos A - \cos B =  -2\sin\left(\frac{A+B}{2}\right)
%			\sin\left(\frac{A-B}{2}\right)$
%	\end{center}
%		\textbf{Law of Sines:} sides: $A$, $B$, and $C$; angles opposite:
%		$\alpha$, $\beta$, and $\gamma$
%	\begin{center}
%		$\frac{A}{\sin \alpha} = \frac{B}{\sin \beta} = \frac{C}{\sin
%			\gamma}$
%	\end{center}
%		\textbf{Law of Cosines:} sides: $A$, $B$, and $C$; angle opposite of
%		$C$ = $\gamma$
%	\begin{center}
%		$C^2 = A^2 + B^2 - 2AB \cos \gamma$
%	\end{center}
%		\textbf{Circular Arclength:} $s = r\theta$; $\theta$ in rad

	\subsection*{Calculus}
	%\subsubsection*{Fundamental theorem of calculus}
	%	\textbf{First part}: Define $F(x) = \int_a^x f(t) dt$ in interval
	%	$[a,b]$ with $f$ continuous and real-valued in $[a,b]$.  Then, $F$ is
	%	continuous on $[a,b]$, differentiable on $(a,b)$, and $F'(x) = f(x) \,
	%	\forall x \in [a,b]$\\
	%	\vspace{4pt}
	%	\textbf{Corollary}: If $f$ is a real-valued continuous function on
	%	$[a,b]$, and $g$ is an antiderivative of $f$ in $[a,b]$, then $\int_a^b
	%	f(x)dx = g(b)-g(a)$.\\
	%	\vspace{4pt}
	%	\textbf{Second part (stronger than corollary)}: Let $f$ be a real-valued
	%	function defined on $[a,b]$ with an antiderivative $g$ on $[a,b]$
	%	(i.e., $f(x)=g'(x) \,\, \forall x \in [a,b]$).  If $f$ is integrable on
	%	$[a,b]$ then $\int_a^bf(x)dx = g(b)-g(a)$.  Note that here $f$ needn't
	%	be continuous.

		\subsubsection*{Basic theorems}
			\textbf{Chain rule} $\frac{\mathrm dy}{\mathrm dx}=\frac{\mathrm dy}{\mathrm du}\cdot\frac
				{\mathrm du}{\mathrm dx}$ \\
			\textbf{Product rule} $\dfrac{d}{dx}(u\cdot v)=u\cdot \dfrac{dv}{dx}+v\cdot \dfrac{du}{dx}$ \\
			\textbf{Integration by parts} $\int u\, \frac{dv}{dx}\; dx=uv-\int v\, \frac{du}{dx} \; dx$

		\subsubsection*{Calculus of variations}
			$J = \int_{x_1}^{x_2} f\{y(x), y\prime (x); x\}$\\
			$\frac{\partial f}{\partial y}-\frac{d}{dx}\frac{\partial
			f}{\partial y\prime} = 0$\\
			$\frac{\partial f}{\partial x}-\frac{d}{dx}\left(f-y\prime
			\frac{\partial f}{\partial y\prime}\right) = 0$\\
			$f-y\prime \frac{\partial f}{\partial y\prime} =
				const$ for $\frac{\partial f}{\partial x} = 0$

	\subsection*{Vectors}
	\textbf{Divergence theorem (Gauss' thm):}
		$ \int_{V}(\nabla\cdot \bm{F}) \, d_{V} =
			\oint_S \bm{F} \cdot \bm{n} \, dS $\\
			%\oint_S \bm{F} \cdot \bm{n} \, dS $\\
	\textbf{Stokes theorem:}
	$ \int_S \nabla \times \bm{F} \cdot dS = \oint_C \bm{F}\cdot d\bm{r} $\\
	\textbf{Dot product:} $ \bm{A} \cdot \bm{B} = |A||B|\cos\theta\bm{\hat{n}} = A_xB_x+A_yB_y+A_zB_z$ \\
	\textbf{Cross product:}$ \bm{A} \times \bm{B} = |A||B|\sin\theta \bm{\hat{n}}  $\\


	\subsection*{Coordinate systems \& conversions}
	\begin{center}
	\textbf{Coordinate Conversion}\\
	\vspace{2.5pt}
	\begin{tabular}{ c !{\color{black}\vline} c !{\color{black}\vline} c }
		\setlength\arrayrulewidth{1pt}
		\textit{cartesian to\dots} & \textit{cylindrical to\dots} &
			\textit{spherical to\dots} \\
		\cline{1-1}\cline{2-2}\cline{3-3}
		%\hhline{-::-::-}
		%\hline
		\textbf{cylindrical}    & \textbf{cartesian} & \textbf{cartesian} \\
		\cline{1-1}\cline{2-2}\cline{3-3}
		%\hline
		$\rho=\sqrt{x^2+y^2}$ & $x=\rho\cos\phi$ & $x=r\sin\theta\cos\phi$ \\
		$\phi=\arctan(y/x)$    & $y=\rho\sin\phi$ & $y=r\sin\theta\sin\phi$ \\
		$z = z$               & $z=z$            & $z=r\cos\theta$\\
		\cline{1-1}\cline{2-2}\cline{3-3}
		\textbf{spherical}      & \textbf{spherical} & \textbf{cylindrical} \\
		\cline{1-1}\cline{2-2}\cline{3-3}
		%\hline
		$r=\sqrt{x^2+y^2+z^2}$& $r=\sqrt{\rho^2+z^2}$&$\rho=r\sin\theta$ \\
		$\theta=\arccos(z/r)$  & $\theta=\arctan(\rho/z)$&$\phi=r\sin\theta$ \\
		$\phi=\arctan(y/x)$    & $\phi=\phi$      & $z=r\cos\theta$ \\
	\end{tabular}
	\vspace{5pt}\\

	\textbf{Unit Vector Conversion}\\
	\vspace{2.5pt}
	\begin{tabular}{ @{}c@{} !{\color{black}\vline} c !{\color{black}\vline} @{}c@{} }
	%\begin{tabular}{ c | c | c }
		\textit{cartesian to\dots} & \textit{cylindrical to\dots} &
			\textit{spherical to\dots} \\
		%\vspace{.5pt}\\
		\hline
		%\vspace{.5pt}\\
		\textbf{cylindrical}    & \textbf{cartesian} & \textbf{cartesian} \\
		%\vspace{.5pt}\\
		\hline
		%\vspace{.5pt}\\

		$ \bm{\hat{\rho}} = \frac{x}{\rho}\bm{\hat{x}} +
				\frac{y}{\rho}\bm{\hat{y}} $ &
		$ \bm{\hat{x}} = cos\phi\bm{\hat{\rho}} - sin\phi\bm{\hat{\phi}} $ &
		\vspace{-8pt}
			\parbox[t]{.35\columnwidth}{\vspace{-16.5pt}
			\begin{equation*}\begin{split}
				&\bm{\hat{x}} =
					\sin\theta\cos\phi\bm{\hat{r}} + \\
					&\cos\theta\cos\phi\bm{\hat{\theta}} -
					\sin\phi\bm{\hat{\phi}}
			\end{split}\end{equation*} } \\

		$ \bm{\hat{\phi}} = -\frac{y}{\rho}\bm{\hat{x}} + 
			\frac{x}{\rho}\bm{\hat{y}} $ &
		$ \bm{\hat{x}}=cos\phi\bm{\hat{\rho}}-sin\phi\bm{\hat{\phi}} $ &
		\vspace{-8pt}
		\parbox[t]{.35\columnwidth}{\vspace{-16.5pt}
			\begin{equation*}\begin{split}
				&\bm{\hat{y}} = \sin\theta\sin\phi\bm{\hat{r}} + \\
					&\cos\theta\sin\phi\bm{\hat{\theta}} +
					\cos\phi\bm{\hat{\phi}}
				\end{split}\end{equation*} } \\

		$ \bm{\hat{z}} = \bm{\hat{z}} $ &
		$ \bm{\hat{z}} = \bm{\hat{z}} $ &
		$ \bm{\hat{z}} = \cos\theta\bm{\hat{r}}-\sin\theta\bm{\hat{\theta}} $\\

		%\vspace{.5pt}\\
		\hline
		%\vspace{.5pt}\\
		\textbf{spherical}      & \textbf{spherical} & \textbf{cylindrical} \\
		%\vspace{.5pt}\\
		\hline
		%\vspace{.5pt}\\

		$ \bm{\hat{r}} = \frac{x\bm{\hat{x}} + y\bm{\hat{y}} +
			z\bm{\hat{z}}}{r} $ &
		$ \bm{\hat{r}} = \frac{\rho}{r}\bm{\hat{\rho}} + 
			\frac{z}{r}\bm{\hat{z}} $ &
		$ \bm{\hat{\rho}} = \sin\theta\bm{\hat{r}} +
			\cos\theta\bm{\hat{\theta}} $ \\

		$ \bm{\hat{\theta}} = \frac{xz\bm{\hat{x}} + yz\bm{\hat{y}} -
			\rho^2\bm{\hat{z}}} {r\rho} $ &
		$ \bm{\hat{\theta}} = \frac{z}{r}\bm{\hat{\rho}} -
			\frac{\rho}{r}\bm{\hat{z}} $ &
		$ \bm{\hat{\phi}} = \bm{\hat{\phi}} $ \\

		$ \bm{\hat{\phi}} = \frac{-y\bm{\hat{x}} + x\bm{\hat{y}}}{\rho} $ &
		$ \bm{\hat{\phi}} = \bm{\hat{\phi}} $ &
		$ \bm{\hat{z}} = \cos\theta\bm{\hat{r}} - \sin\theta\bm{\hat{\theta}}$\\
		
	\end{tabular}\\
	\end{center}
	
	\begin{center}
		\textbf{Differential Elements}\\
		\vspace{2.5pt}
		%\begin{tabular}{ c | c | c }
		\begin{tabular}{ @{}c@{} !{\color{black}\vline} c !{\color{black}\vline} @{}c@{} }
			\textbf{cartesian} & \textbf{cylindrical} & \textbf{spherical} \\

			\hline

			\parbox[t]{.290\columnwidth}{\vspace{-10pt}
				\begin{equation*}\begin{split}
					d\bm{\bm{l}} = dx\bm{\hat{x}} + dy\bm{\hat{y}} +& \\
						dz\bm{\hat{z}}
				\end{split}\end{equation*} } &

			\parbox[t]{.295\columnwidth}{\vspace{-10pt}
				\begin{equation*}\begin{split}
					d\bm{\bm{l}} = d\rho\bm{\hat{\rho}} +
						\rho d\phi\bm{\hat{\phi}} +& \\
						dz\bm{\hat{z}}
				\end{split}\end{equation*} } &

			\parbox[t]{.335\columnwidth}{\vspace{-10pt}
				\begin{equation*}\begin{split}
					d\bm{\bm{l}} = dr\bm{\hat{r}} +
						rd\theta\bm{\hat{\theta}} + & \\
						r\sin\theta d\phi\bm{\hat{\phi}}
				\end{split}\end{equation*} } \\

			\hline

			\parbox[t]{.290\columnwidth}{\vspace{-10pt}
				\begin{equation*}\begin{split}
					d\bm{\bm{A}} =& dy\, dx\, \bm{\hat{x}} + \\
						&dx\, dz\, \bm{\hat{y}} + \\
						&dx\, dy\, \bm{\hat{z}}
				\end{split}\end{equation*} } &

			\parbox[t]{.295\columnwidth}{\vspace{-10pt}
				\begin{equation*}\begin{split}
					d\bm{\bm{A}} =& \rho d\phi\, dz\, \bm{\hat{\rho}} + \\
						&d\rho\, dz\, \bm{\hat{\phi}} + \\
						&\rho\, d\rho\, d\phi\, \bm{\hat{z}}
				\end{split}\end{equation*} } &

			\parbox[t]{.335\columnwidth}{\vspace{-10pt}
				\begin{equation*}\begin{split}
					d\bm{\bm{A}} =& r^2\sin\theta \, d\theta \, d\phi \,
							\bm{\hat{r}} + \\
						&r\sin\theta\, dr \, d\phi \, \bm{\hat{\theta}} + \\
						& r \, dr \, d\theta \, \bm{\hat{\phi}}
				\end{split}\end{equation*} } \\

			\hline

			\parbox[t]{.290\columnwidth}{\vspace{-10pt}
				\begin{equation*}\begin{split}
					dV =& dx \, dy \, dz
				\end{split}\end{equation*} } &

			\parbox[t]{.295\columnwidth}{\vspace{-10pt}
				\begin{equation*}\begin{split}
					dV =& \rho \, d\rho \, d\phi \, dz
				\end{split}\end{equation*} } &

			\parbox[t]{.335\columnwidth}{\vspace{-10pt}
				\begin{equation*}\begin{split}
					dV =& r^2 \, \sin\theta \, dr \, d\theta \, d\phi
				\end{split}\end{equation*} } \\
		\end{tabular}\\
	\end{center}

	\begin{center}
		\textbf{POSITIONS, VELOCITIES, \& ACCELERATIONS}\\
	\end{center}
	\begin{mydescription}
	  \item[\textbf{polar}] \ \\
		$\bm{\bm{r}} = \rho\bm{\bm{e_\rho}}$\\
		$\bm{\bm{v}} = \dot{\rho}\bm{\bm{e_\rho}} + \rho\dot\theta\bm{\bm{e_\theta}}$\\
		$\bm{\bm{a}} = \left(\ddot{\rho}-\rho\dot\theta^2\right)\bm{\bm{e_\rho}} + 
		\left( \rho\ddot\theta+2\dot{\rho}\dot\theta\right)\bm{\bm{e_\theta}}$\\
	\item[\textbf{cylindrical}] \ \\
		$\bm{\bm{r}} = \rho\bm{\bm{e_\rho}} + z\bm{\bm{e_z}}$\\
		$\bm{\bm{v}} = \dot{\rho}\bm{\bm{e_\rho}} + \rho\dot\theta\bm{\bm{e_\theta}} + \dot{z}\bm{\bm{e_z}}$\\
		$\bm{\bm{a}} = \left(\ddot{\rho}-\rho\dot\theta^2\right)\bm{\bm{e_\rho}} + 
		\left( \rho\ddot\theta+2\dot{\rho}\dot\theta\right)\bm{\bm{e_\theta}} +
		\ddot{z}\bm{\bm{e_z}}$

	\item[\textbf{spherical}] \ \\
		$\bm{\bm{r}} = \rho \bm{\bm{e_\rho}}$\\
		$\bm{\bm v} = \dot \rho \bm{\bm{e_\rho}} + \rho\dot \theta \bm{\bm{e_\theta}} +
			\rho \dot\phi\sin\theta\bm{\bm{e_\phi}}$\\
		$\bm{\bm{a}} = \left( \ddot \rho -
			\rho\dot\theta^2-\rho\dot\phi^2\sin^2\theta\right)\bm{\bm{e_\rho}} +
			\left(\rho\ddot\theta+2\dot \rho \dot\theta -
			\rho\dot\phi^2\sin\theta\cos\theta\right)\bm{\bm{e_\theta}} + \left(
			\rho\ddot\phi\sin\theta+2\dot \rho\dot\phi\sin\theta +
			2\rho\dot\theta\dot\phi\cos\theta\right)\bm{\bm{e_\phi}}$
	\end{mydescription}
	\begin{center}\textbf{del, $\bm{\nabla}$, in CARTESIAN}\end{center}
		\begin{mydescription}
			\item[del operator:]
				$\bm{\nabla} =
				\bm{e_x} \frac{\partial}{\partial x} +
				\bm{e_y}\frac{\partial}{\partial y} + 
				\bm{e_z}\frac{\partial}{\partial z}$  
			\item[gradient:]
				$\bm{\nabla}\phi =
				grad\,\phi =
				\bm{e_x}\frac{\partial\phi}{\partial x} +
				\bm{e_y}\frac{\partial\phi}{\partial y} +
				\bm{e_z}\frac{\partial\phi}{\partial z}$  
			\item[directional derivative:]
				$\frac{d\phi}{ds} =
					\bm{\nabla}\phi \cdot \frac{\bm{A}}{|\bm{A}|}$  
			\item[divergence:]
				$\bm{\nabla}\cdot\bm{V} =
				div\, \bm{V} =
					\frac{\partial V_x}{\partial x} +
					\frac{\partial V_y}{\partial y} +
					\frac{\partial V_z}{\partial z}$  
			\item[curl:]
				$\bm{\nabla}\times \bm{V} =
				\bm{e_x}\left(\frac{\partial V_z}{\partial y} -
					\frac{\partial V_y}{\partial z}\right) +
					\bm{e_y} \left( \frac{\partial V_x}{\partial z} -
					\frac{\partial V_z}{\partial x}\right) +
					\bm{e_z} \left( \frac{\partial V_y}{\partial x} -
					\frac{\partial V_x}{\partial y} \right) $  
			\item[Laplacian:]
				$\Delta f = \bm{\nabla}^2\phi =
				\bm{\nabla} \cdot (\bm{\nabla}\phi) = div\, grad\, \phi =
					\frac{\partial^2 \phi}{\partial x^2} +
					\frac{\partial^2 \phi}{\partial y^2} +
					\frac{\partial^2 \phi}{\partial z^2}$ 
		\end{mydescription}

		\begin{center}\textbf{del, $\bm{\nabla}$, in CYLINDRICAL}\end{center}
		\begin{mydescription}
			\item[gradient:]
				$\bm{\nabla}f =
				grad\,f =
				\frac{\partial f}{\partial\rho}\bm{e_\rho} +
				\frac{1}{\rho}\frac{\partial f}{\partial\phi}\bm{e_\phi} +
				\frac{\partial f}{\partial z}\bm{e_z}$
			\item[divergence:]
				$\bm{\nabla}\cdot\bm{V} =
				div\, \bm{V} =
				\frac{1}{\rho}\frac{\partial(\rho\bm{V}_\rho)}{\partial\rho} +
				\frac{1}{\rho}\frac{\partial \bm{V}_\phi}{\partial\phi} +
				\frac{\partial \bm{V}_z}{\partial z}
				$
			\item[curl:]
				$\bm{\nabla}\times \bm{V} =
				\left({ \frac{1}{\rho}\frac{\partial V_z}{\partial\phi} -
					\frac{\partial V_\phi}{\partial z}}\right)\bm{e_\rho}+
				\left({ \frac{\partial V_\rho}{\partial z} -
					\frac{\partial  V_z}{\partial\rho}}\right)\bm{e_\phi}+
				\frac{1}{\rho}\left({
					\frac{\partial(\rho V_\phi)}{\partial\rho} -
					\frac{\partial V_\rho}{\partial\phi}}\right)\bm{e_z}
				$
			\item[Laplacian:]
				$\Delta f = \bm{\nabla}^2f =
				\bm{\nabla} \cdot (\bm{\nabla}f) =
				\frac{1}{\rho}\frac{\partial}{\partial\rho}\left({
				\rho\frac{\partial f}{\partial\rho}}\right) +
				\frac{1}{\rho^2}\frac{\partial^2 f}{\partial\phi^2} +
				\frac{\partial^2 f}{\partial z^2}
				$
		\end{mydescription}
		
		\begin{center}\textbf{del, $\bm{\nabla}$, in SPHERICAL}\end{center}
		\begin{mydescription}
			\item[gradient:]
				$\bm{\nabla}f = grad\,f =
				\frac{\partial f}{\partial r}\bm{e_r} +
				\frac{1}{r}\frac{\partial f}{\partial \theta}\bm{e_\theta} +
				\frac{1}{r\sin\theta}\frac{\partial f}{\partial\phi}\bm{e_\phi}
				$
			\item[divergence:]
				$\bm{\nabla}\cdot\bm{V} = div\, \bm{V} =
				\frac{1}{r^2}\frac{\partial(r^2V_r)}{\partial r} +
				\frac{1}{r\sin\theta}\frac{\partial}{\partial\theta}\left({
				V_\theta\sin\theta}\right) +
				\frac{1}{r\sin\theta}\frac{\partial V_\phi}{\partial\phi}
				$
			\item[curl:]
				$\bm{\nabla}\times \bm{V} =
				\frac{1}{r\sin\theta}\left({ \frac{\partial}{\partial\theta}(V_\phi\sin\theta)-\frac{\partial V_\theta}{\partial\phi} }\right)\bm{e_r} +
				\frac{1}{r}\left({\frac{1}{\sin\theta}\frac{\partial V_r}{\partial\phi} - \frac{\partial}{\partial r}(r V_\phi) }\right) \bm{e_\theta} +
				\frac{1}{r}\left({\frac{\partial}{\partial r}(r V_\theta) -
				\frac{\partial V_r}{\partial\theta} }\right) \bm{e_\phi}
				$
			\item[Laplacian:]
				$\Delta f = \bm{\nabla}^2f =
				\bm{\nabla} \cdot (\bm{\nabla}f) =
				\frac{1}{r^2}\frac{\partial}{\partial r}\left( 
				r^2\frac{\partial f}{\partial r}\right) +
				\frac{1}{r^2\sin\theta}\frac{\partial}{\partial\theta}\left(
				sin\theta\frac{\partial f}{\partial\theta}\right) +
				\frac{1}{r^2\sin^2\theta}\frac{\partial^2f}{\partial\phi^2}
				$
		\end{mydescription}
		

%	\subsection*{Fourier series}
%		\begin{mydescription}
%			\item[real-valued functions, period of $2l$:]
%			$ f(x) = \frac{a_0}{2} + \sum_{n=1}^\infty \left(
%					a_n\cos \frac{n\pi x}{l} +
%					b_n \sin \frac{n \pi x}{l}\right) $
%			\begin{eqnarray*}
%				a_0 &=& \frac{1}{l}\int_{-l}^{l} \! f(x) \, dx\\
%				a_n &=& \frac{1}{l}\int_{-l}^{l} \!
%					f(x) cos \frac{n\pi x}{l} \, dx, \,\, n \geq 1\\
%				b_n &=& \frac{1}{l}\int_{-l}^{l} \!
%					f(x)sin \frac{n\pi x}{l} \, dx, \,\, n \geq 1\\
%			\end{eqnarray*}
%			\item[complex-valued functions,]
%				period of $2l$: \\
%				$ f(x) = \sum_{n=-\infty}^{\infty} c_n e^{in\pi x/l} $
%				$$
%					c_n = \frac{1}{2l}\int_{-l}^{l} \! f(x)e^{-in\pi x/l} \, dx,
%						\,\, n \in \mathbb{Z}
%				$$
%			\item[convergence]
%				(Dirichlet): 
%				If $f(x)$ is periodic of period $2 l$, and if between $-l$ and
%				$l$ it is single-valued, has a finite number of max. and min.
%				values, and a finite number of discont., and if
%				$\int_{-l}^{l} \, |f(x)| \, dx$ is finite, Fourier series
%				converges to $f(x)$ at all points where $f(x)$ is continuous.
%				At discontinuities, series converges to midpoint of the jump.\\
%		\end{mydescription}
   
	\subsection*{Taylor series}
		Taylor series of $f(x)$ about $x=a$:\\
		\[ f(x) = f(a) + (x-a)f'(a) + \frac{1}{2!}(x-a)^2f''(a) + \cdots +
				\frac{1}{n!}(x-a)^nf^{(n)}(a) + \cdots \]
	
%	\subsection*{Green's method}
%		$$
%			x(t) = \int_{-\infty}^{t}F(t')G(t,t')dt'
%		$$
	
	\subsection*{Ordinary differential equations}
		\subsubsection*{Separable 1\textsuperscript{st}-order}
			Equation can be written as $f(y)dy = f(x)dx$, such as
			$\frac{dy}{dx} = N(1-y)$.  Evaluate integrals directly.

		\subsubsection*{Linear 1\textsuperscript{st}-order}%: \\
			Write the equation in the form $ y' + P(x)y = Q(x) $ and then
			define\\
			{\centering $ I = \int P(x)dx $\\}
			and find $y$ by solving \\
			{\centering $ ye^{I} = \int Q(x)e^{I}dx+c $\\}

		%\subsubsection*{Homogeneous \textital{Functions in}:
			%1\textsuperscript{st}-Order}:(Note that this is NOT the same
			%as a %Homogeneous) \underline{Homogeneous \textital{functions}
			%(NOT diff. eq's}: Write the equation in the form

		\subsubsection*{Linear 2\textsuperscript{nd}-order homogeneous with
				constant coefficients}
			Equations of the form\\
			{\centering $a_2 \frac{d^2y}{dx^2} + a_1 \frac{dy}{dx} + a_0y = 0$\\}
			Write the characteristic polynomial $a_2D^2y + a_1 Dy + a_0y = 0$
			and factor into $(D-a)(D-b)y = 0$.  In general, this can be solved
			by letting $u=(D-a)y$, solving the 1\textsuperscript{st}-order diff
			eq $(D-b)u=0$ for $u(x)$, substituting this solution into the
			equation $(D-a)y=u(x)$, and finally solving \emph{this} linear
			1\textsuperscript{st}-order ODE.  In fact, this method can be
			generalized to higher-order linear diff eq's.  However, there are
			pre-determined solution forms based upon the relationships between
			$a$ and $b$:\\
			{\centering $ a, b \in \mathbb{R}, a \neq b \Rightarrow
			y=c_1e^{ax}+c_2e^{bx} $\\}
			{\centering $ a, b \in \mathbb{R}, a = b \Rightarrow y=(Ax+B)e^{ax} $\\}
			For\\
			{\centering $ a, b \in \mathbb{C}, a = b^\ast = \alpha \pm i\beta, $\\}
			any of the following forms are solutions:\\
			\begin{gather*}
				y=Ae^{\alpha + i\beta x} + Be^{\alpha - i\beta x}\\
				y=e^{\alpha x}\left(Ae^{i\beta x} + Be^{-i\beta x}\right)\\
				y=e^{\alpha x}\left(c_1\sin\beta x + c_2\cos\beta x\right)\\
				y=ce^{\alpha x}\sin\left( \beta x + \gamma \right)\\
				y=ce^{\alpha x}\cos\left( \beta x + \delta \right)
			\end{gather*}

		\subsubsection*{Linear 2\textsuperscript{nd}-order inhomogeneous
				with constant coefficients}
			Equations of one of the forms\\
			$$
				a_2 \frac{d^2y}{dx^2} + a_1 \frac{dy}{dx} + a_0y = f(x)
			$$
			$$
				\frac{d^2y}{dx^2} + \frac{a_1}{a_2} \frac{dy}{dx} +
					\frac{a_0}{a_2}y = F(x)
			$$
			can be solved, generally, as described for the homogeneous case,
			but with $F(x)$ on the right-hand side when solving the first
			1\textsuperscript{st}-order ODE, $(D-b)u=F(x)$.  (This gives both
			the particular \emph{and} complementary solution.)  Otherwise, find
			$y = y_c + y_p$ where $y_c$, the complementary solution, comes from
			solving the homogeneous equation and $y_p$ is a particular solution
			from a pre-computed form for specific $F(x)$:\\
			\begin{center}
			$(D-a)(D-b)y=F(x)=ke^{cx}$, particular solution $y_p$ is given by:\\
			\begin{tabular}{ l l }
				$y_p=Ce^{cx}$ & if $c$ is not equal to either $a$ or $b$; \\
				$y_p=Cxe^{cx}$ & if $c$ equals $a$ or $b$, $a\neq b$;\\
				$y_p=Cx^2e^{cx}$ & if $c=a=b$ \\
			\end{tabular}\\
			\emph{(For $F(x)=k\cos\alpha x$ or $F(x)=k\sin\alpha x$, solve the
			above with $F(x)=ke^{c=i\alpha x}$ and take the real or imag part,
			respectively. For $F(x)=const$, set $c=0$.)}
			\end{center}
			A more general form of this (called the \emph{method of
			undetermined coefficients}) follows:\\
			\begin{center}
				$(D-a)(D-b)y=F(x)=e^{cx}P_n(x)$; $P_n(x)$ is a polynomial of
				degree $n$:
				\[
				y_p = 
				\begin{cases}
					   e^{cx}Q_n(x) & \text{if $c \neq a$ and $c \neq b$}\\
					  xe^{cx}Q_n(x) & \text{if $c=a$ or $c=b$, $a\neq b$}\\
					x^2e^{cx}Q_n(x) & \text{if $c=a=b$}
				\end{cases}
				\]
			\end{center}

%\section*{Constants}
%	\begin{myitemize}
%		\item[$c = $] 
%		  speed of light in vacuum $ = 2.998\times 10^8$ m$/$s
%		\item[$\mu_0 = $]
%			mag const / perm of vacuum $ = 4 \pi \times 10^{-7}$
%			N$\cdot$A$^{-2}$ or H$\>$m$^{-1}$
%		\item[$\varepsilon_0 = $]
%			elec const / permit of vacuum $ = 8.854 \times 10^{-12}$
%			F$\>$m$^{-1}$
%		\item[$Z_0 = $]
%			char impedance of vacuum $ = 376.73$ $\Omega$
%		\item[$h = $]
%		  Planck's const $ = 6.626 \times 10^{-34}$ J $\cdot$ s
%		  $ = 4.136 \times 10^{-15}$ eV$\cdot$ s
%
%		\vspace{4pt}
%
%		\item[$e = $]
%		  charge of electron  $ = 1.602\times10^{-19}$ C
%		\item[$m_\mathrm{e} = $]
%		  mass of electron $ = 9.109\times10^{-31}$ kg $ = 0.511$ MeV/$c^2$
%		\item[$m_\mathrm{n} = $]
%		  mass of neutron or proton $ = 1.67\times10^{-27}$ kg
%		  $ = 938$ MeV/$c^2$
%		\item[$\mu_\mathrm{B} = $]
%		  Bohr magneton, $e\hbar/2m_\mathrm{e} = 9.274\times10^{-24}$ J/T
%		  $ = 5.7884\times10^{-4}$ eV/T
%		\item[$\mu_N = $]
%		  Nuclear magneton, $e\hbar/2m_\mathrm{p} = 5.051\times10^{-27}$ J/T
%
%		\vspace{4pt}
%
%		\item[$G = $]
%		  gravit. constant $ = 6.674\times10^{-11}$ N$\>$m$^2$kg$^{-2}$
%		\item[$g = $]
%		  gravit. accel on Earth surface $ = 9.8 \>$m$/$s$^{2}$
%
%		\vspace{4pt}
%
%	  	\item[$R_\mathrm{S} = $]
%		  mean radius of Sun $ = 696\times10^{6}$ m
%		\item[$R_\mathrm{E} = $]
%		  mean radius of Earth $ = 6.371\times10^{6}$ m
%		\item[$R_\mathrm{M} = $]
%		  mean radius of Moon $ = 1.737\times10^{6}$ m
%
%		\vspace{4pt}
%
%	  	\item[$R_\mathrm{S,E} = $]
%		  mean distance, Earth to Sun $ = 149.6\times10^{9}$ m
%		\item[$R_\mathrm{M,E} = $]
%		  mean distance, Earth to Moon $ = 384.4\times10^{6}$ m
%
%		\vspace{4pt}
%
%	  	\item[$M_\mathrm{S} = $]
%		  mass of Sun $ = 1.99\times10^{30}$ kg
%		\item[$M_\mathrm{E} = $]
%		  mass of Earth $ = 5.98\times10^{24}$ kg
%		\item[$M_\mathrm{M} = $]
%		  mass of Moon $ = 7.35\times10^{22}$ kg
%
%		\vspace{4pt}
%
%	  	\item[$k_\mathrm{B} = $]
%		  Boltzmann's constant $ = 1.38 \times 10^{-23} \>$J/K
%		\item[$R = $]
%		  Ideal gas constant $ = 8.315 \>$J/mol$\cdot$K
%		\item[$N_A = $]
%		  Avogadro's number $ = 6.02214179 \times 10^{23} \>$mol$^{-1}$
%
%	\end{myitemize}

%\section*{Unit Conversions}
%	\begin{myitemize}
%	    \item[Distance] \ \\
%		  %$\mathrm{mL}=\mathrm{cm^3}$ \\
%		  \AA $= 1 \times 10^{-10} \: \mathrm{m} $
%	    \item[Area] \ \\
%	    \item[Volume] \ \\
%		  $\mathrm{mL}=\mathrm{cm^3}$ \\
%		  $\mathrm{L}=1\times10^{-3}\:\mathrm{m^3}$
%		\item[Velocity] \ \\
%		\item[Mass] \ \\
%		  $\mathrm{u = 1.661 \times 10^{-27}\:kg}$
%		\item[Pressure] \ \\
%		  $\mathrm{Pa = N/m^2}$ \\
%		  $\mathrm{atm} = 1.013 \times 10^{5}\:\mathrm{N/m^2}$ \\
%		  $\mathrm{atm} = 1.013 \mathrm{bar}$ \\
%		  $\mathrm{atm} = 14.7\;\mathrm{lb/in^2}$ \\
%		  $\mathrm{atm} = 760\;\mathrm{mm\,Hg}$
%		\item[Energy] \ \\
%		  $\mathrm{J = kg \cdot m^2 / s^2 = N \cdot m}$ \\
%		  $\mathrm{eV = 1.602 \times 10^{-19}\:J}$ \\
%		  $\mathrm{Btu = 1054 \:J}$ \\
%		  $\mathrm{cal = 4.186 \:J}$ \\
%		  $\mathrm{Cal = 1\,000 \:cal}$
%		\item[Power] \ \\
%		  $\mathrm{W = J/s}$
%		\item[Force] \ \\
%		  $\mathrm{N = kg \cdot m/s^2}$
%		\item[Temp] \ \\
%		  $\mathrm{^\circ R} = \frac{5}{9}\mathrm{\:K}$ \\
%		  $\mathrm{^\circ C} = T[\mathrm{K}]-273.15$ \\
%		  $\mathrm{^\circ F} = \frac{9}{5}T[\mathrm{C}]+32$
%		\item[STP] \ \\
%		  $300\;\mathrm{K}$ and $1\;\mathrm{atm}$
%	\end{myitemize}

\section*{Classical Mechanics}

	\subsection*{Newton's laws}
		\begin{mydescription}
			\item[1\textsuperscript{st}:]
				Body remains at rest or in uniform motion unless acted upon by
				a force  \\
			\item[2\textsuperscript{nd}:]
				$\bm{F}_{tot} = \frac{d\bm{p}}{dt} = m\bm{a}$	\\
			\item[3\textsuperscript{rd}:]
				$\bm{F}_{A \rightarrow B} = -\bm{F}_{B \rightarrow A}$\\
		\end{mydescription}
	
	\subsection*{Lagrangian dynamics}
			  \textbf{Hamilton's principle} --- Nature minimizes (makes
			  stationary) the action. \\
			  \textbf{Constrained} --- If a 3D system of $N$ particles has $n <
			  3N$ minimum generalized coordinates, the system is
			  \emph{constrained}. \\
			  \textbf{Natural} --- The coordinates $q_n$ are \emph{natural} if
			  the relationships of $r_\alpha$ (every particle's position) to
			  $q_n$ doesn't change with time.\\
			  \textbf{Ignorable} --- a coordinate $q_i$ is \emph{ignorable} if
			  the corresponding generalized momentum $p_i$ is constant.\\
			  \textbf{Lyupanov Stability} --- If $x_e$ is Lyapunov stable
			  and all solutions that start out near $x_e$ converge to $x_e$,
			  then $x_e$ is asymptotically stable
		\begin{mydescription}
			\item[Lagrangian:]
			  $\mathscr{L}=T-U$
			\item[Action:]
			  $S = \int_{t_1}^{t_2}\mathscr{L}(q_1,q_2,\,\dots\, ,q_N,\dot{q}_1,\dot{q}_2,\,\dots \,,\dot{q}_N,t)dt$
			\item[Euler-Lagrange equations:]$\frac{\partial \mathscr{L}}{\partial q_1}=\frac{d}{dt}\frac{\partial \mathscr{L}}{\partial \dot{q_1}},\,\dots$ etc.
			\item[Generalized forces:]$F_i=\frac{\partial \mathscr{L}}{\partial
			  q_i}$
			\item[Generalized momenta]$p_i = \frac{\partial \mathscr{L}}{\partial \dot{q_i}}$
		\end{mydescription}

	\subsection*{Orbits}
		\begin{mydescription}
		  \item[Definitions:] \ \\
			$M=m_1+m_2$ \\
			$\mu=\frac{m_1m_2}{m_1+m_2}$; e.g., $U(\rho)=\frac{-Gm_1m_2}{\rho}$ \\
			$\bm{r}$: vector from body 1 to body 2 \\
			$\bm{R}$: vector from origin in inertial frame to system's CM
		  \item[Kinetic energy:] $T=\frac{1}{2}M\dot{r}^2+\frac{1}{2}\mu\dot{r}^2$
		  \item[Lagrangian:] $\mathscr{L}=\frac{1}{2}\mu\dot{\rho}^2+\frac{1}{2}\mu{\rho}^2\dot{\phi}^2-U(\rho)$
		  \item[Solution in $\phi$:] $\dot\phi=\frac{\ell}{\mu\rho^2}$ ($\ell$ const --- angular momentum)
		  \item[Solution in $\rho$:] $\mu\ddot{\rho}=-\frac{d}{d\rho}U(\rho) + \frac{\ell^2}{\mu\rho^3} = -\frac{d}{d\rho}\left[U(\rho)+\frac{\ell^2}{2\mu\rho^2}\right]$
		  \item[Effective potential:]$U_{eff}=U(\rho)+\frac{\ell^2}{2\mu\rho^2}$
		  \item[Note cons. of energy:] $\frac{d}{dt}\left( \frac{1}{2}\mu\dot\rho^2 \right) = -\frac{d}{dt}U_{eff}(\rho)$; $E=\frac{1}{2}\mu\dot\rho^2+U_{eff}(\rho)$
		  \item[Use:]$u=1/r$ and $\frac{d}{dt}=\frac{d\phi}{dt}\frac{d}{d\phi}=\dot\phi\frac{d}{d\phi}=\frac{\ell}{\mu\rho^2}\frac{d}{d\phi}=\frac{\ell u^2}{\mu}\frac{d}{d\phi}$
		  \item[$u$-equation:]$u''(\phi)=-u(\phi)-\frac{\mu}{\ell^2u(\phi)^2F(u)}$
		  \item[Use:]$\gamma=Gm_1m_2$ and $F(u)=-\gamma u^2$; then $U''(\phi)=-u(\phi)+\gamma\mu/\ell^2$; use $w(\phi)=u(\phi)-\gamma\mu/\ell^2$, so $W(\phi)=A\cos(\phi-\delta)$ ergo $u(\phi)=\frac{\gamma\mu}{\ell^2}+A\cos\phi$
		  \item[Radial eqn:] $r(\phi)=\frac{r_c}{1+\varepsilon\cos\phi}$
		  \item[Cartesian:] $\left( \frac{x+\frac{r_c\varepsilon}{1-\varepsilon^2}}{\frac{r_c}{1-\varepsilon^2}} \right)^2+\left( \frac{y}{\frac{r_c}{\sqrt{1-\varepsilon^2}}} \right)^2$
		  \item[Eccentricity:] $\varepsilon=A\cdot r_c$ ($A$ some constant)
		  \item[Circular orbit:] $r_c=\ell^2/\gamma\mu$
		  \item[Min radius:] $r_{min}=\frac{r_c}{1+\varepsilon}=\frac{\ell^2}{\gamma\mu(1+\varepsilon)}$ (at $\phi=0$; \textbf{perihelion}); $\ell=\mu rv_{tan}$
		  \item[Max radius:] $r_{max}=\frac{r_c}{1-\varepsilon}$ (at $\phi=\pi$; \textbf{aphelion})
		  \item[Radial velocity:]$v_r = \sqrt{\frac{\mu}{r_c}}\cdot\varepsilon\cdot\sin\phi$
		  \item[Tangential velocity:]$v_t = \sqrt{\frac{\mu}{r_c}}\cdot\left(1+\varepsilon\cdot\cos\phi\right)$
		  \item[Ellipse params:] $a=\frac{r_c}{1-\varepsilon^2}$; $b=\frac{r_c}{\sqrt{1-\varepsilon^2}}$; $d=a\varepsilon$; $\varepsilon=\sqrt{1-(b/a)^2}$
		  \item[Orbital period:] $\tau=2\pi\cdot a\cdot\sqrt{\frac{a}{\mu}}$
		  \item[Energy:] $E=\frac{\gamma^2\mu}{2\ell^2}(\varepsilon^2-1)$
		  \item[Kepler's $1^{st}$ law:] Orbits: ellipses w/ sun at a focus (approx. true)
		  \item[Kepler's $2^{nd}$ law:] Line from Sun to planet, const. area/time\\
			$dA=\frac{1}{2}r^2d\phi$; $\frac{dA}{dt}=\frac{1}{2}\frac{\ell}{\mu}$, inep. of time
		  \item[Kepler's $3^{rd}$ law:] $\tau=\frac{A}{dA/dt}=\frac{2\pi ab\mu}{\ell} \Rightarrow \tau^2=4\pi^2\frac{a^3r_c\mu^2}{\ell^2}=4\pi^2\frac{a^3\mu}{\gamma}\approx \frac{4\pi^2}{GM_s}a^3$
		\end{mydescription}


	\subsection*{Rigid Body Dynamics}
		\begin{mydescription}
		  	\item[center of mass:] \ \\
			  $\bm{R} = \frac{1}{M}\sum_{\alpha=1}^Nm_\alpha\bm{r_\alpha}$ (discrete point masses)\\
			  $\bm{R} = \frac{1}{M}\int\bm{r}\; dm$ (continuous mass distr.)
			\item[momentum:] $\bm{p}\equiv m\bm{v}$
			\item[kinetic energy:] \ \\
			  $T_{tot} = T\mathrm{(motion\;of\;CM)}+T\mathrm{(rotation\;about\;CM)}$ \\
			  $T_{tot} = T_{rot}\mathrm{(about\;an\;instantaneously\;fixed\;point\;in\;body)}$ \\
	    	  $T_{trans} = \frac{1}{2}m|\bm{v}|^2$  \\
			  $T_{rot} = \frac{1}{2}\bm{\omega}\cdot\bm{L}$ \\
			  $T_{rot} = \frac{1}{2}\left( \lambda_1\omega_1^2 +
			  \lambda_2\omega_2^2 +  \lambda_3\omega_3^2 + \right)$ (if coord.
			  sys = principal axes)
			  $T_{rot} = \frac{1}{2}I\omega$ (freshman physics model)
			\item[moment of inertia, point mass:]
				$I = \int r^2dm$ or, for a point mass, $I = r^2m$, where $r$ is
				the perp. distance to axis of rotation  \\
			\item[moment of inertia, rigid body:]
			  $\mathbb{I} = \int \begin{pmatrix} y^2+z^2 & -xy & -xz \\ -xy &
				x^2+z^2 & -yz \\ -yz & -xz & x^2+y^2 \end{pmatrix} dM $ \\
			  \item[principal axes:] Any axis through $O$ with
				$\bm{\omega}\parallel\bm{L}$ when $\bm{\omega}$
				points along that axis; i.e.,
				$\bm{L}=\lambda\bm{\omega}$. Principal axes are
				eigenvectors of $\mathbb{I}$. \\
			\item[parallel axis theorem:]
				$I_z=I_{cm}+md^2$; $I_{cm}$: inertia about center of mass, $m$:
				mass, $d$: distance between axes  \\
			\item[angular momentum:] \ \\
			  $\bm{L}\equiv \bm{r} \times \bm{p}$ ($\bm{r}$: position vec, $\bm{p}$: linear momentum)   \\
			  $\bm{L} = \mathbb{I}\vec{\bm{\omega}}$ \\
			  $\bm{L} = \bm{L}\mathrm{(motion\;of\;CM)}+\bm{L}\mathrm{(motion\;relative\;to\;CM)}$
			\item[torque:]
				$\bm{N}\equiv \bm{r}\times \bm{F} = \dot{\bm{L}} =
				\bm{r}\times\dot{\bm{p}}$  \\
			\item[work:]
				$W = N \theta$, $\theta$ in rad \\
			\item[angle:]
				$\bm{\theta}$  \\
			\item[angular velocity:]
				$\bm{\omega} =
				\dot{\bm{\theta}} =
					\frac{\bm{r}\times \bm{v}}{|\bm{r}|^2}$\\
			\item[linear velocity:]
				$\bm{v}=\bm{\omega}\times \bm{r}$ \\
			\item[angular acceleration:]
				$\bm{\alpha} =
					\dot{\bm{\omega}} = \ddot{\bm{\theta}} = \bm{a}_T/r$;
					$\bm{a}_T$ is tangential acceleration  \\
			\item[newton's 2\textsuperscript{nd}-law:]
					$\bm{N} = I\bm{\alpha}$
			\item[time deriv, unit vec in rotating frame:]
			  $\frac{d\bm{e}}{dt}=\bm{\omega}\times\bm{e}$ ($\bm{e}$ fixed in
			  body)
			\item[time deriv, vec in rotating frame:]$\left( \frac{d\bm{r}}{dt}
			  \right)_{S_0} = \left( \frac{d\bm{r}}{dt} \right)_{S} +
			  \bm{\omega} \times \bm{r}\,$ ($S_0$: inertial frame, $S$: rotating frame)
			\item[Newton's $2^{nd}$ in rotating frame:] $m\ddot{\bm{r}} =
			  \bm{F}+2m\dot{\bm{r}}\times\bm{\omega}+m\left(
			  \bm{\omega\times\bm{r}} \right)\times\bm{\omega}$
			\item[Euler's equations of motion (\emph{body frame}):]
			  $\dot{\bm{L}}+\bm{\omega}\times\bm{L}=\bm\tau$ \dots \\
			  $ \lambda_1\dot\omega_1-\left( \lambda_2-\lambda_3 \right)\omega_2\omega_3 = \tau_1 $ \\
			  $ \lambda_2\dot\omega_2-\left( \lambda_3-\lambda_1 \right)\omega_3\omega_1 = \tau_2 $ \\
			  $ \lambda_3\dot\omega_3-\left( \lambda_1-\lambda_2 \right)\omega_1\omega_2 = \tau_3 $
			\item[stability:] If $\lambda_1<\lambda_2<\lambda_3$, rotations
			  about $\bm{\hat{e}_1}$ and $\bm{\hat{e}_3}$ are stable, while
			  rotations about  $\bm{\hat{e}_2}$ are not. If
			  $\lambda_1=\lambda_2$, rotations about all principal axes are
			  stable.
		\end{mydescription}

	\subsection*{Coupled Oscillators}
		\begin{mydescription}
		  \item[] $\bm{M\ddot x = -Kx}$, \hspace{5pt} with
			$(\bm{M,K})\in\mathbb{R}^{N\times N}$ and
			$\bm{x}\in\mathbb{R}^{N\times 1}$
		  \item[assume solution:] $\bm{x} = \Re \left\{
			\bm{z}(t)=\bm{a}_ne^{i\left(\omega_nt-\delta_n\right)} \right\}$\\
			$\bm{a}_n\in\mathbb{R}^{N\times 1}$ = eigenvectors \\
			$\omega_n\in\mathbb{R}$ = eigenvalues \\
			$\delta_n\in\mathbb{R}$ = phase term (can be excluded, whereupon
			  $a_n\in\mathbb{C}$) \\
		  \item[actual solution:]
			$\bm{x}=\Re\left\{\sum_nA_n\bm{a}_ne^{i\left(\omega_nt-\delta_n\right)}\right\}$,
			$A_n\in\mathbb{R}$
		  \item[normal frequencies:] $\omega_n$, $n\in[1,N]$ where $N$ is
			dimension of $K$; generalized eigenvalues of system.
		  \item[normal modes:] solutions to equations of motion with only one
			of $\left\{ \omega_n \right\}$; \emph{all} motion can be described
			as a weighted sum of the normal modes; equations of motion written
			in terms of $\xi_n$ diagonalize both $\bm{M}$ and $\bm{K}$
		  \item[normal coordinates:] vary independently of one another \\
			e.g.: 2 $m$'s, 3 $k$'s, $k_1=k_3$: use
			$\xi_1=\frac{1}{2}\left( x_1+x_2 \right)$ and
			$\xi_2=\frac{1}{2}\left( x_1-x_2 \right)$ 
		\end{mydescription}
		
	
	\subsection*{Conservative Force}
		\begin{mydescription}
			\item[\emph{Conditions, given $\bm{F}$ has continuous first partials in
			  a simply connected region\dots}]
			\item[No curl anywhere:]
				$\bm{\nabla}\times\bm{F}=0$
			\item[Equal work regardless of path]: \\
			  $W_C = \int_{C}\bm{F} \cdot d\bm{s} = const. \;\forall\;\mathrm{paths}\;C$\\
				$W_C = \oint_{C}\bm{F} \cdot d\bm{s} = 0 \;\forall\;\mathrm{closed\;contours}\;C$
			\item[$\bm{F} \cdot d\bm{r}$ is exact differential]
			\item[$\bm{F} = \nabla W$, $W$ single-valued]
			\item[Allows definition of potential:]
			  $\bm{F} = - \nabla \bm{U}$
		\end{mydescription}
	\subsection*{Specific Forces}
		\begin{mydescription}
			\item[gravity] \  \\
				point mass or sph.-symm mass:
				$\bm{F} = -G \, {\frac{ M \, m}{r^2}}\bm{e_r} \approx -mg$
				on earth \\
				generally:
				$\bm{F} = -Gm \int_V \frac{\rho(\bm{r}')\bm{e_r}}{r^2}dv'$ \\
				grav field vector: $\bm{g} \equiv - \bm{\nabla} \Phi
					= \bm{F}/m$ \\
				grav potential, point mass: $\Phi = -G\frac{M}{r}$\\
				grav potential, mass distr: $\Phi = -G\int_V
					\frac{\rho(\bm{r}')}{r}dv'$\\
				potential energy: $U = m \Phi$\\
				Gauss' law for grav, int: $\oint_S \bm{g}\cdot \, d\bm{A} = -4\pi GM$\\
				Gauss' law for grav, dif: $\nabla \cdot \bm{g}=-4\pi G\rho$\\
				Poisson's equation: $\nabla^2\phi = 4\pi G\rho$, for rad-sym
				system, this is $\frac{1}{r^2}\frac{\partial}{\partial
				r}\left(r^2\frac{\partial\phi}{\partial r}\right) = 4\pi G
				\rho(r)$ and $\bm{g}(r)=-\bm{e_r}
				\frac{\partial\phi}{\partial r}$

			\vspace{4pt}

			\item[tidal] (due to Moon's gravity) \\
				$\bm{e_R}$ points from Moon's center to test mass on Earth \\
				$\bm{e_D}$ is from center of Moon to center of Earth \\
				$(x,y)$ ECEF coord of test mass\\
				$\bm{F}_T=-GmM_m \left( \frac{\bm{e_R}}{R^2} -
				\frac{\bm{e_D}}{D^2}\right)$\\
				$F_{T_x} \approx 2GmM_mx/D^3$\\
				$F_{T_y} \approx -GmM_my/D^3$

		\item[spring] (simple, linear) \\
			$F = -kx$ ($x$: displ from eq lib, $k$: spring const)  \\
			$U = \frac{1}{2}kx^2$\\

		\item[inertial force, linear accel:] $\bm{F_{inert}}=-m\bm{A}$
			($\bm{A}$: frame's accel w.r.t. inertial frame)
		\item[centrifugal] (inertial force) \\
		  $\bm{F_{centr}}=m\left( \bm{\Omega}\times\bm{r}
		  \right)\times\bm{\Omega}$ (generally)\\
		  $\bm{F_{centr}} = \frac{mv^2}{r}\bm{e_r} =
		   mr\Omega^2\bm{e_r}$ (for circular motion)\\
		  $U_{centr}(r) = \frac{\ell^2}{2mr^2}$ ($\ell$: angular momentum)\\
		  Free-fall accel (e.g., on Earth): $\bm{g}=\bm{g}_0+\left(
		  \bm{\Omega\times\bm{R}} \right)\times\bm{\Omega}$

		\item[coriolis] (inertial force)\\
		  $\bm{F_{cor}}=2m\dot{\bm{r}}\times \bm{\Omega}$

		\item[friction] (pseudo-force) \\
			$\bm{F}_f = \mu F_N$ ($\mu$: static ($\mu_s$) or
			kinetic ($\mu_k$), $F_N$: normal force)\\
			Angle of friction (obj starts to move): $tan \theta = \mu_s$\\
			Energy converted to heat: $E_{th} = \mu_k \int F_n(x)dx$

		\item[general retarding] \  \\
			$\bm{F} = -bm\dot{\bm{x}}^n$ ($b$: damping const,
			$m$: mass, $n$: power of velocity dep., just 1 in simple cases)  \\

		\item[air resistance / drag] \  \\
			$W=\frac{1}{2}c_W\rho Av^2$,
			$c_W$: dimensionless drag coeff, $\rho$: air density,
			$A$: cross-sectional area perp. to velocity ($v$) \\

		\item[buoyant] \  \\
			$F = \rho_{fluid}Vg$, dir. opposite to grav.-induced pressure grad.
			in fluid; $\rho_{fluid}$: density, $V$: submerged volume,
			$g$: grav. \\

		\item[lorentz] \  \\
			$\bm{F} = q\bm{v}\times\bm{B}$,
			$q$ is charge of particle, $\bm{v}$ its velocity,
			$\bm{B}$ is mag. field strength  \\

		\item[electrostatic] \  \\
			$\bm{F} = q\bm{E}$, $q$ is charge, $\bm{E}$ electric field
		\end{mydescription}
	
	\subsection*{Energy}
		\begin{mydescription}
			\item[potential energy:]
				$\displaystyle\int_1^2\bm{F}\cdot d\bm{r} \equiv U_1 - U_2$\\
				(work, done by force $\bm{F}$, req'd to move particle from
				point 1 to point 2 with no change in kinetic energy); potential
				energy is the capacity to do work.  \\
			\item[force due to the potential $U$:]
				$\bm{F} = -\bm{\nabla}U$  \\
			\item[kinetic energy:]
			    $T_{trans} \equiv \frac{1}{2}m|\bm{v}|^2$  \\
				$T_{rot} \equiv \frac{1}{2}\bm{\omega}\cdot\bm{L}$  \\
				$T = \frac{\bm{p}^2}{2m}$  \\
			\item[total energy:]
				$E \equiv T + U$  \\
			\item[1D solution given $E$ and $U(x)$,]
				for conservative force only:\\
				$$
					t-t_0 = \int_{x_0}^x\frac{\pm dx}
					{\sqrt{\frac{2}{m}\left[E-U(x)\right]}}
				$$
		\end{mydescription}

		%\item[potential energy:]
		%	$\int_1^2\vec{f}\cdot d\vec{r} \equiv U_1 - U_2$ (work, done by
		%	force $\vec{F}$, req'd to move particle from point 1 to point 2
		%	with no change in kinetic energy); potential energy is the capacity
		%	to do work.  \\
		%\item[force due to the potential $U$:]
		%	$\vec{F} = -\vec{\nabla}U$  \\
		%\item[kinetic energy:]
		%	$T \equiv \frac{1}{2}mv^2$  \\
		%\item[total energy:]
		%	$E \equiv T + U$  \\
		%\item[1D solution given $E$ and $U(x)$], for conservative force
		%only:\\
		%	$$
		%		t-t_0 = \int_{x_0}^x\frac{\pm dx}
		%		{\sqrt{\frac{2}{m}\left[E-U(x)\right]}}
		%	$$
	
	\subsection*{Conservation theorems}
		\begin{mydescription}
			\item[linear momentum:]
				$\frac{d}{dt}\left(p_1 + p_2\right) = 0$ (or $p_1+p_2$ is const)
				if no external forces act upon system  \\
			\item[angular momentum:]
				$\dot{\bm{L}}=\bm{r}\times\dot{\bm{p}}=0$ (or $\bm{L}$ is
				const) if no external torque acts upon system \\
			\item[energy:]
				$\bm{F}+\nabla U = 0$; $\frac{dE}{dt} = 0$ if the force field
				represented by $\bm{F}$ is conservative
		\end{mydescription}
	
%    \subsection*{Harmonic oscillation}
%    \subsubsection*{Simple harmonic oscillator}
%    	{\centering
%    	$m\ddot{x} = -kx$  \\
%    	$\omega_0^2 \equiv k/m$  \\
%    	$\ddot{x} + \omega_0^2x = 0$  \\
%    	$x(t) = A\sin (\omega_0 t - \delta)$  \\
%    	$E = T + U = \frac{1}{2}kA^2$ \\
%    
%    	}
%    
%    \subsubsection*{Damped oscillator}
%    	\begin{mydescription}
%    		\item[equation of motion:]
%    			$m\ddot{x} + b\dot{x} + kx = 0$; $b$ is resisting force coeff,
%    			$k$ is restoring force coeff  \\
%    		\item[convenient substitutions:]
%    			$\beta \equiv \frac{b}{2m}$ (damping), and $\omega_0^2 \equiv
%    			k/m$ (natural ang. freq, undamped sys)  \\
%    		\item[new eqn of motion:]
%    			$\ddot{x} + 2\beta\dot{x} + \omega_0^2x = 0$  \\
%    		\item[general sol'n:]
%    			$$
%    				x(t) = e^{-\beta t}\left[ A_1exp\left(
%    				\sqrt{\beta^2-\omega_0^2}t\right) + A_2exp\left(
%    				-\sqrt{\beta^2-\omega_0^2}t\right) \right]
%    			$$
%    		\item[underdamping:]
%    			$\omega_0^2 > \beta^2$  \\
%    		\item[critical damping:]
%    			$\omega_0^2 = \beta^2$  \\
%    		\item[overdamping:]
%    			$\omega_0^2 < \beta^2$  \\
%    	\end{mydescription}
%    
%    \subsubsection*{Sinusoidally-driven damped oscillator}
%    	\begin{mydescription}
%    		\item[eqn of motion:]
%    			$m\ddot{x} + b\dot{x} + kx = F_0\cos\omega t$;
%    			$b$ is resisting force coeff, $k$ is restoring force coeff  \\
%    		\item[convenient substitutions:]
%    			$A=F_0/m$ (driving ampl), $\beta \equiv \frac{b}{2m}$
%    			(damping), and $\omega_0^2 \equiv k/m$ (natural ang. freq,
%    			undamped sys)  \\
%    		\item[new eqn of motion:]
%    			$\ddot{x} + 2\beta\dot{x} + \omega_0^2x = A\cos\omega t$  \\
%    		\item[complementary solution:]
%    			$ x_c(t) = e^{-\beta t}\left[ A_1exp\left(
%    				\sqrt{\beta^2-\omega_0^2}t\right) + A_2exp\left(
%    				-\sqrt{\beta^2-\omega_0^2}t\right) \right] $\\
%    		\item[particular solution:] .\\
%    			$ x_p(t)=\frac{A}{\sqrt{\left(
%    				\omega_0^2-\omega^2\right)^2+4\omega^2\beta^2}} cos(\omega
%    				t-\delta) $\\
%    			$ \delta = \arctan\left( \frac{2\omega\beta}{\omega_0^2 -
%    				\omega^2} \right) $
%    		\item[amplitude resonance frequency:]
%    			$\omega_R = \sqrt{\omega_0^2-2\beta^2} \,\, (\omega_r <
%    			\omega_1 < \omega_0)$ \\
%    		\item[kinetic energy resonance frequency:]
%    			$\omega_E = \omega_0$ \\
%    		\item[quality factor:]
%    			$Q \equiv \frac{\omega_R}{2\beta} \approx
%    			\frac{\omega_0}{\Delta\omega}$ (the latter is for lightly
%    			damped systems; $\Delta\omega$ is the distance between
%    			half-energy points --- $D_res/\sqrt{2}$ --- on the amplitude
%    			resonance curve) \\
%    	\end{mydescription}
%    
%    \subsubsection*{Underdamped oscillator}
%    	\begin{mydescription}
%    		\item[pseudo-frequency of oscillation:]
%    			$\omega_1^2 \equiv \omega_0^2 - \beta^2$  \\
%    		\item[solution (form 1):]
%    			$x(t) = e^{-\beta t}\left[ A_1e^{i\omega_1t} +
%    			A_2e^{-i\omega_1t} \right]$  \\
%    		\item[solution (form 2):]
%    			$x(t) = Ae^{-\beta t}\cos(\omega_1t-\delta)$  \\
%    		\item[phase plot:]
%    			Use the var. subst. $u=\omega_1 x$, $w=\beta x+\dot{x}$ and
%    			plot $w$ on the $y$-axis vs. $u$ on the $x$-axis \\
%    		\item[response to $\delta$ force:]
%    			$x(t) = \frac{b}{\omega_1}e^{-\beta
%    			(t-t_0)}\sin\omega_1(t-t_0)$  \\
%    		\item[green's fn:]
%    			$G(t,t') \equiv \frac{1}{m\omega_1}e^{-\beta
%    			(t-t')}\sin\omega_1(t-t')$, $t\geq t'$; $0$ otherwise  \\
%    	\end{mydescription}
%    
%    \subsubsection*{Critically damped oscillator}
%    	\begin{mydescription}
%    		\item[qualitative behavior:]
%    			System approaches equilibrium (natural solution dies out)
%    			faster than the others.\\
%    		\item[solution:]
%    			$x(t) = (A+Bt)e^{-\beta t}$  \\
%    	\end{mydescription}
%    
%    \subsubsection*{Overdamped oscillator}
%    	\begin{mydescription}
%    		\item[pseudo-frequency of (non-)oscillation:]
%    			$\omega_2^2 \equiv \beta^2 - \omega_0^2$  \\
%    		\item[solution:]
%    			$x(t) = e^{-\beta t}\left[ A_1e^{\omega_2t} + A_2e^{-\omega_2t}
%    			\right]$  \\
%    		\item[phase plot:]
%    			Asymptotic behavior tends towards $\dot{x}=-(\beta-\omega_2)x$
%    			unless $A_1=0$, then it goes to $\dot{x}=-(\beta+\omega_2)x$
%    	\end{mydescription}
%    
%    %\subsubsection*{Dirac Delta-Driven Underdamped Oscillator}
%    %	\item[response to $\delta$ force:]
%    %		$x(t) = \frac{b}{\omega_1}e^{-\beta (t-t_0)}\sin\omega_1(t-t_0)$  \\
%    %	\item[green's fn:]
%    %		$G(t,t') \equiv \frac{1}{m\omega_1}e^{-\beta
%    %		(t-t')}\sin\omega_1(t-t')$, $t\geq t'$; $0$ otherwise  \\
%    
%    \subsubsection*{Series RLC circuit}
%    	\begin{mydescription}
%    		\item[voltage across inductor:]
%    			$V_L = L \frac{dI}{dt} = L\ddot{q}$  \\
%    		\item[voltage across resistor:]
%    			$V_R = IR = R\frac{dq}{dt} = R\dot{q}$  \\
%    		\item[voltage across capacitor:]
%    			$V_C = \frac{q}{C}$  \\
%    		\item[diffeq of RLC circuit with driving power source:]
%    			$L\ddot{q} + R\dot{q} + q/C = V(t)$  \\
%    	\end{mydescription}
    
%    \subsection*{Electrical--mechanical equivalents}
%    	\begin{tabular}{l l l l}
%    		& \textbf{Mechanical} & & \textbf{Electrical} \\[3pt]
%    		\\[3pt]
%    		$x$       & Displacement            & $q$           & Charge \\
%    		[3pt]
%    		$\dot{x}$ & Velocity                & $\dot{q}=I$   & Current \\
%    		[3pt]
%    		$m$		  & Mass                    & $L$           & Inductance \\
%    		[3pt]
%    		$b$		  & Damping resistance 		& $R$			& Resistance \\
%    		[3pt]
%    		$l/k$	  & Mechanical compliance 	& $C$			& Capacitance \\
%    		[3pt]
%    		$F$		  & Ampl of impressed force & $\varepsilon$ &
%    											Ampl of impressed emf
%    	\end{tabular}

%\section*{Thermodynamics \& Statistical Mechanics}
%  \subsubsection*{Fundamentals}
%    \textbf{0$^{th}$ Law}---If sys $A$ in therm. eqlib w/ sys $B$ and $B$
%    	in therm eqlib w/ sys $C$, $A$ in eqlib w/ $C$ \\
%    	\textbf{1$^{st}$ Law}---Cons. of energy; $\Delta U_{tot} = Q+W_{tot}$ \\
%    \textbf{2$^{nd}$ Law}---$S_{tot}$ for an isolated sytem always stays
%    	the same or increases \\
%    \textbf{3$^{rd}$ Law}---As $T \rightarrow 0$, $S \rightarrow 0$; OR as
%    	$T\rightarrow0$, $C_V\rightarrow0$ \\
%    \textbf{Fundamental Assumption of Stat Mech}---Given an isolated system
%    	in equilibrium, it is found with equal probability in each of its
%    	accessible microstates \\
%  \subsubsection*{Definitions}
%    \textbf{Temperature}---$T\equiv \left( \frac{\partial S}{\partial U} \right)^{-1}$\\
%    1. Measure of the tendency of an object to spontaneously give up energy to
%      its surroundings. \\
%    2. That which is the same for two systems in thermal equilibrium. \\
%    \emph{NOTE} A negative temperature is \emph{hotter} than an infinite
%    temperature. \\
%	\vspace{2.5pt}
%  \textbf{Heat}---Any spontaneous flow of energy from one object to another
%    caused by a difference in temp between the objects. \\
%	\emph{Mechanisms}: conduction, convection, radiation \\
%	\vspace{2.5pt}
%  \textbf{Work}---Any other transfer of energy into or out of a system.\\
%    \hspace{5pt}\emph{quasistatic compression}:
%    $W=-P\Delta V = -\int_{V_i}^{V_f}P(V)\:\textrm{d}V$ \\
%  \vspace{2.5pt}
%	\textbf{Heat Capacity}---Heat required to increase the temp of a substance \\
%	  \hspace{5pt} $\mathbb{C} \equiv \frac{Q}{\Delta T}$\\
%	  \hspace{5pt} specific heat capacity: $c = \frac{Q}{m \Delta T} = \frac{\mathbb{C}}{m}$\\
%	  \hspace{5pt} $\mathbb{C}_V = \left( \frac{\partial U}{\partial T} \right)_{N,V}$
%	    (since $Q=\Delta U-W$ and $W=-P\Delta V$); ``energy capacity''\\
%	  \hspace{5pt} $\mathbb{C}_P = \left( \frac{\partial U}{\partial T} \right)_{P}+P\left(
%	    \frac{\partial V}{\partial V} \right)_P = \left( \frac{\partial H}
%		{\partial T} \right)_P$; ``enthalpy capacity''\\
%  \vspace{2.5pt}
%  \textbf{Latent Heat}---For a phase transition (1$^{st}$-order), energy goes
%    into / comes out of molecular rearrangement, \emph{not} a change in KE, so $Q
%	\nRightarrow \Delta U$, and $\mathbb{C}\rightarrow \infty$. \\
%	\hspace{5pt} $L \equiv \frac{Q}{m}$ \\
%  \vspace{2.5pt}
%  \textbf{Entropy}---$S=k\ln \Omega$ [J/K] \\
%    \hspace{5pt} $\Delta S = \Delta U/T = Q/T$ (const. temp \& vol, no work)\\
%	\hspace{5pt} $dS = Q/T =\frac{C_VdT}{T} \Rightarrow \Delta S = S_f-S_i = \int_{T_i}^{T_f} \frac{C_V}{T}dT$ ($V$ const, $W$=0, $T$ varying)\\
%	\hspace{5pt} Mixing \emph{different} gasses: $\Delta S_{A}=N_{A}\ln(V_{f,A}/V_{i,A})$; $\Delta S_{tot} = \Delta S_A + \Delta S_B$ \\
%
%    \hspace{5pt} Einstein solid, high-T: $S \approx Nk\ln U - Nk\ln (\epsilon N) + Nk$ \\
%	\hspace{5pt} Monatomic ideal gas: $S \approx Nk\ln V + NK\ln U^{3/2} + f(N)$ \\
%  \vspace{2.5pt}
%  \textbf{Enthalpy}---$H$; $\Delta H$ is equal to the change in the internal
%    energy of the system, plus the work that the system has done on its
%    surroundings. \\
%	\hspace{5pt} $H \equiv U + P \mathrm{d}V$ \\
%	\hspace{5pt} $\Delta H = \Delta U + P \Delta V + W_{non-compr/exp}$ \\
%	\hspace{5pt} If $P$ is const., $\Delta H = Q + W_{non-compr/exp}$ \\
%  \vspace{2.5pt}
%  \textbf{Gibbs Free Energy}---$G \equiv U+PV-TS = H-TS$ \\
%    \hspace{5pt} Energy $TS$ into rabbit from surroundings; extra energy $PV$ to displace atmosphere \\
%  \vspace{2.5pt}
%  \textbf{Helmholtz Free Energy}---$F \equiv U - TS$ \\
%    \hspace{5pt} Energy $TS$ into rabbit from surroundings; NO accounting for $PV$ to displace atmosphere \\
%    \hspace{5pt} $ F = -kT \ln Z $\\
%    \hspace{5pt} $ S = - \left( \frac{\partial F}{\partial T} \right){V,N} $\\
%    \hspace{5pt} $ P = - \left( \frac{\partial F}{\partial V} \right){T,N} $\\
%    \hspace{5pt} $ \mu = + \left( \frac{\partial F}{\partial N} \right){T,V} $\\
%    \vspace{2.5pt}
%  \textbf{Ideal Gas Law}---$PV=Nk_{B}T$ \\
%    \hspace{5pt} Note that a non-ideal gas will reduce pressure on vessel due to interactions\\
%  \subsubsection*{Equipartition Theorem}
%    $U_{\textrm{therm}}=Nf\frac{1}{2}kT$ \\
%    Thermal energy distributes evenly to each quadratic degree of freedom
%    (true even for relativistic systems, but \emph{not} true for
%    quantum-dominated systems). \\
%    \textbf{Diatomic gas at low temp} has 3 DOF (3 transl) \\
%    \textbf{Diatomic gas at room temp} has 5 DOF (3 transl + 2 rot) \\
%    \textbf{Diatomic gas at high temp} has 5 DOF (3 transl + 2 rot + 2 vibr) \\
%    \textbf{Einstein solid} has 6 DOF (3 springs = 3$\times$(1 PE + 1 KE)) \\
%    \textbf{Liquid} has 3 quad DOF and other DOF non-quadratic \\
%  \subsubsection*{Sackur-Tetrode equation}
%    $S=NK\left[ ln\left( \frac{V}{N}\left( \frac{4\pi mU}{3Nh^2} \right)^{3/2} \right)+\frac{5}{2} \right]$ (Valid for ideal, monatomic gas)
%  \subsubsection*{Isothermal Compression} (Quasistatic)
%    $\Delta T = 0$, or $T = const.$; slow heat exchange w/ outside equalizes temp\\
%	$\Rightarrow \Delta U = 0$\\
%	$\Rightarrow Q = -W$\\
%	For ideal gas:\\
%	\hspace{5pt}$W = -\int_{V_i}^{V_f}PdV = -\int_{V_i}^{V_f}NkT/V\;dV = -NkT \ln (Vf/Vi)$\\
%	\hspace{5pt}$\Rightarrow W>0\;$ if $\;V_i > V_f$\\
%	\hspace{5pt}$\Rightarrow W<0\;$ if $\;V_i < V_f$\\
%  \subsubsection*{Adiabatic Compression} (Quasistatic)
%    $Q=0$ (Isolated system can't exchange heat)\\
%	$\Rightarrow \Delta U = W$\\
%	For ideal gas:\\
%	\hspace{5pt}$\Delta U = -P \Delta V$ \\
%	\hspace{5pt}$Nf\frac{1}{2}kdT = -P \Delta V$; since $P=NkV/T\;$, $\;Nf\frac{1}{2}kdT = -\frac{NkT}{V}dT$\\
%	\hspace{5pt}$T^{f/2}V=const.$\\
%	\hspace{5pt}$PV^{(f+2)/f}=const.$\\
%	\hspace{5pt}Define $\gamma = (f+2)/f$, and $PV^\gamma=const.$
%  \subsubsection*{Heat Conduction Law (Fourier)} 
%  \subsubsection*{Probabilities and Multiplicities} 
%  \textbf{Combinations:} \\
%  \hspace{5pt} $\textrm{comb}(A,B) =\frac{A!}{A!B!}$ or ``$A$ choose $B$'' \\
%  \textbf{$N$ coins, multiplicity of macrostate with $n$ heads:}\\
%	\hspace{5pt}$\Omega(N,n)=\frac{N!}{n!(N-n)!}= comb(N,n)$ \\
%	\textbf{Einstein solid w/ $N$ osc ($N/3$ atoms) and $q$ units of energy ($hf$):} \\
%	\hspace{5pt} $\Omega(N,q)=\frac{(q+N-1)!}{q!(N-1)!} = comb(q+N-1, q)$ \\
%	\textbf{2 Einstein solids w/ $N_A$, $N_B$, and $q_{tot}$:} \\
%	\hspace{5pt} Solid $A$ can have $q_A \in [0,q_{tot}$ or $q_{tot}+1$ different energy levels \\
%	\hspace{5pt} Solid $B$ has $q_B=q_{tot}-q_A$ energy \\
%	\hspace{5pt} Multiplicity for a given macrostate (def'd by $q_A$): $\Omega_{tot} = \Omega_A \Omega_B$ \\
%	\hspace{5pt} Number of possible microstates: $\Omega_{grand}=\Omega(N_A+N_B,q_{tot)}$ \\
%	\hspace{5pt} Prob of microstate: $\mathbb{P} = \Omega_{tot}/\Omega_{grand}$ \\
%	\hspace{5pt} Most prob. microstate: ${q_A} = \frac{N_A}{N_A+N_B}q_{tot}$ \\
%	\textbf{52 cards, there are:} \\
%	\hspace{5pt} $comb(52,5)$ possible hands \\
%	\hspace{5pt} $10(4^5-4)$ straights \\
%	\hspace{5pt} $10(4^5)$ straights + straight flushes \\
%	\hspace{5pt} $40$ straight flushes \\
%	\hspace{5pt} $13\cdot12\cdot comb(4,2)\cdot comb(4,3)$ full houses \\
%	\textbf{Stirling's Approximation (for $N \gg 1$):} \\
%	\hspace{5pt} $N! \approx N^Ne^{-N}\sqrt{2\pi N}$ (strong) \\
%	\hspace{5pt} $N! \approx N^Ne^{-N}$ (weak) \\
%	\hspace{5pt} $\ln (N!) \approx N\ln N - N$ (log of weak) \\
%	\textbf{For $|x| \ll 1$, $\ln (1+x) \approx x$} \\
%
%	\textbf{Multipicities under various approximations:} \\
%	\hspace{5pt} \emph{Einstein solid, large \& high temp ($N\gg 1$; $q\gg N$):} \\
%	\hspace{10pt} $\Omega(N,q) \approx \left( \frac{eq}{N} \right)^N$ \\
%	\hspace{5pt} \emph{Einstein solid, large \& low(er) temp ($N\gg 1$; $q\gg 1$):} \\
%	\hspace{10pt} $\Omega(N,q) \approx \left( \frac{eN}{q} \right)^q$ \\
%	\hspace{5pt} \emph{Einstein solid any large $N$ and $q$:} \\
%	\hspace{10pt} $\Omega(N,q) \approx \frac{\left( \frac{q+N}{q} \right)^q \left( \frac{q+N}{N} \right)^N}{\sqrt{2\pi q\left( q+N \right)/N}}$ \\
%	\hspace{5pt} \emph{2 Einstein solids, large \& high temp:} \\
%	\hspace{10pt} $\Omega_{tot} \approx \left( \frac{eq_A}{N_A} \right)^{N_A} \left( \frac{eq_B}{N_B} \right)^{N_B}$; if $N_A = N_B$,  $\Omega_{tot} \approx (e/N)^{2N}(q_A q_B)^N$\\
%	\hspace{5pt} \emph{2-state paramagnet, large ($N_{\uparrow} \ll N$):} \\
%	\hspace{10pt} $\Omega(N,N_{\uparrow}) \approx \left( \frac{eN}{N_{\uparrow}} \right)^{N_{\uparrow}}$ \\
%	\hspace{5pt} \emph{Ideal Gas, $d$-Dimensional:} \\
%	\hspace{10pt} $\Omega_N = \frac{1}{N!} \frac{L^{dN}}{h^{dN}} \frac{2\pi^{dN/2}}{\left( dN/2 - 1 \right)!} \left( 2mU \right)^{\left( dN-1 \right)/2}$ \\
%	\hspace{10pt} ($d$=\# dimensions, $L$=length)  \\
%	\hspace{5pt} \emph{Ideal Gas, 3D:} \\
%	\hspace{10pt} $\Omega_N \approx \frac{1}{N!} \frac{V^{N}}{h^{dN}} \frac{\pi^{3N/2}}{\left( 3N/2 \right)!} \left( 2mU \right)^{3N/2}$ \\
%	\hspace{10pt} $\Omega(U,V,N) \approx f(N) V^N U^{3N/2}$ \\
%	\textbf{Sharpness of multiplicity function:} \\
%
%	\textbf{Boltzmann Statistics:} \\
%	\hspace{5pt} $\mathbb{P}(\textrm{state}=s) = \frac{1}{Z}e^{-E(s)/kT}$; $Z = \sum_s e^{-E(s)/kT}$ \\
%	\hspace{5pt} $\mathbb{P}(\textrm{energy}=s) = \frac{1}{Z}g(s)e^{-E(s)/kT}$; $Z = \sum_s g(s)e^{-E(s)/kT}$ ($g$=degeneracy)\\
%	\hspace{5pt} For non-interacting, indistinguishable particles: $Z_tot = Z_1 \cdot Z_2 \dots Z_N$ \\
%	\hspace{5pt} For non-interacting, distinguishable particles: $Z_{tot} = \frac{1}{N!}Z_1^N$ \\
%	\hspace{5pt} $<E> = \frac{1}{Z}\sum_s E(s) e^{-\beta E(s)}=-\frac{1}{Z}\frac{\partial Z}{\partial\beta}$ ($\beta=1/kT$) \\
%	\hspace{5pt} $\sigma_E = kT\sqrt{C/k}$ \\
%	\hspace{5pt} Maxwell speed distribution: $D(v) = \left( \frac{m}{2\pi kT} \right)^{3/2} 4\pi v^2e^{-mv^2/2kT}$ \\
%	\hspace{5pt} Note from equipartition thm, $v_{rms} = \sqrt{3kT/m}$ for ideal gas \\
%	\hspace{5pt} From Maxwell, $<v> = \sqrt{8kT/\pi m}$ for ideal gas \\
%
%	\subsection*{Thermodynamic Interactions}
%	\textbf{Thermodynamic identity:} $dU = T \:dS - P \:dV + \mu \: dN$ (any
%	  large system) \\
%	  \hspace{0pt} Simplifies to 1$^{st}$ law if $\Delta V$ is quasi-static, no
%	    other forms of work are done, and no other relevant variables are
%	    changed.
%		\begin{tabular}{l l l l}
%		  \textbf{Interaction type}& \textbf{Exch. quant.} & \textbf{Governing var}& \textbf{Formula} \\[3pt]
%			thermal       & energy, $U$ & temp, $T$           & $\frac{1}{T}=\left( \frac{\partial S}{\partial U} \right)_{V,N}$ \\
%			[3pt]
%			mechanical       & volume, $V$ & pressure, $P$       & $\frac{P}{T}=\left( \frac{\partial S}{\partial V} \right)_{U,N}$ \\
%			[3pt]
%			diffusive       & particles, $N$ & chem. pot, $\mu$  & $\frac{\mu}{T}=-\left( \frac{\partial S}{\partial N} \right)_{U,V}$
%		\end{tabular}
%	\subsection*{Heat engines, heat pumps, \& refrigerators}
%	\textbf{Heat engine efficiency:} \\
%	\hspace{5pt} $e \equiv \frac{\textrm{benefit}}{\textrm{cost}} = \frac{W}{Q_h}$ \\
%	\hspace{5pt} since it's a \emph{cycle}, $Q_h = Q_c+W \Rightarrow e = 1-\frac{Q_c}{Q_h}$\\
%	\hspace{5pt} in terms of temp, since 2$^{nd}$ law says $\frac{Q_c}{T_c} \geq \frac{Q_h}{T_h}$, $e \leq 1-\frac{T_c}{T_h}$\\
%	\textbf{Refrigerator efficiency (Coefficient of Performance):} \\
%	\hspace{5pt} $\textrm{COP} \equiv \frac{\textrm{benefit}}{\textrm{cost}} = \frac{Q_c}{W}$ \\
%	\hspace{5pt} since it's a \emph{cycle}, $Q_h = Q_c+W \Rightarrow \textrm{COP} = \frac{1}{Q_h/Q_c-1}$\\
%	\hspace{5pt} in terms of temp, since 2$^{nd}$ law says $\frac{Q_h}{T_h} \geq \frac{Q_c}{T_c}$, $\textrm{COP} \leq \frac{T_c}{T_h-T_c}$\\
%	\textbf{Carnot Cycle:} \\
%	1) isothermal expansion at $T_h$ taking in $Q_h$ \\
%	2) adiabatic expansion to $T_c$ \\
%	3) isothermal compression at $T_c$ expelling $Q_c$ \\
%	4) adiabatic compression to $T_h$ \\
%	\hspace{5pt} $e = 1-\frac{T_c}{T_h}$ ($S$ const for quasi-static adiabatic and isothermic processes) \\
%	\textbf{Otto Cycle:} \\
%	1) \emph{compression}: adiabatic compression of gas (and fuel) in piston \\
%	2) \emph{ignition}: fuel ignited while piston static ($V$ const; $T$ \& $P$ increase) \\
%	3) \emph{power}: adiabatic expansion of gas in cylinder does work \\
%	4) \emph{exhaust}: hot gasses replaced by lower $P$, lower $T$ gas ($V$ const); fuel injected \\
%	\hspace{5pt} $e = 1-\left( \frac{V_2}{V_1}\right) ^{\gamma-1} = 1 - \frac{T_1}{T_2} = 1 - \frac{T_4}{T_3}$; recall $\gamma=(f+2)/f$ \\
%	\textbf{Diesel Cycle:} \\
%	1) \emph{compression}: (isentropic) compression of gas in piston \\
%	2) \emph{injection/ignition}: fuel injected \& ignites; $P$ const while $V\uparrow$ \& (piston moves) \\
%	3) \emph{expansion}: isentropic expansion w/ $P\downarrow$ \\
%	4) \emph{exhaust}: hot gasses replaced by lower $P$, lower $T$ gas ($V$ const) \\
%	\hspace{5pt} $e=1-\frac{1}{r^{\gamma-1}}\left ( \frac{\alpha^{\gamma}-1}{\gamma(\alpha-1)} \right )$; $r=V_1/V_2$=compr. ratio; $\alpha=V_3/V_2$=cut-off ratio


\end{multicols}
}\end{document}
