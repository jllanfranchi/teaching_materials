%  use 
%    xdvi -paper usr formulae &
%  and 
%    dvips -t landscape formulae
%  to preview and make PostScript
%,

\documentclass[letterpaper,landscape,10pt]{article}
\usepackage{multicol}
\usepackage{calc}
\usepackage{ifthen}
\usepackage[landscape]{geometry}
\usepackage{amssymb}
\usepackage{amsmath}
\usepackage{titlesec}
\usepackage{pxfonts}
%\usepackage{millennial}
%\usepackage[charter]{mathdesign}
%\usepackage{fouriernc}
\usepackage{mathrsfs}
\usepackage{bm}

\titleformat{\section}[display]
	{\scshape\normalsize\filcenter}
	{\thesection}
	{1pt}
	{\titlerule \vspace{2pt} \small}
	[\vspace{1pt} \titlerule]
 
\titleformat{\subsection}{\small\sffamily}{\thesubsection}{1em}{}{}{}
\titleformat{\subsubsection}{\scriptsize\slshape}{\thesubsubsection}{1em}{}{}{}

\titlespacing*{\section}      {-10pt}{5pt}{5pt}
\titlespacing*{\subsection}   {-8pt}{10pt}{2pt}
\titlespacing*{\subsubsection}{-4pt}{10pt}{2pt}

\geometry{top=.50in,left=.50in,right=.50in,bottom=.50in}
\title{Equations, Exam 2.  Physics 2210, Classical Mechanics 1. P1}
\author{Justin Lanfranchi}
\date{2009.10.31}
\renewcommand{\baselinestretch}{.5}
\newif\iftechexplorer\techexplorerfalse
\markright{Phys 2210, Classical Mechanics \hrulefill\ Exam 2 equation sheet,
	Page }


\newenvironment{mydescription}
{\begin{description}
	\setlength{\itemsep}{0pt}
	\setlength{\parskip}{0pt}
	\setlength{\parsep}{-1pt}}
{\end{description}}

\newenvironment{myitemize}
{\begin{itemize}
	\setlength{\itemsep}{-1pt}
	\setlength{\parskip}{0pt}
	\setlength{\parsep}{0pt}}
{\end{itemize}}


\begin{document}{
\raggedright

\fontsize{7}{1}\selectfont
%\pagestyle{headings}
\begin{multicols}{3}
%\setlength{\premulticols}{18pt}
%\setlength{\postmulticols}{18pt}
%\setlength{\multicolsep}{18pt}
%\setlength{\columnsep}{18pt}
\iftechexplorer
  \maketitle
\fi

\section*{Mathematics}

	%\subsection*{Quadratic equation}
	%	Solution to $ax^2+bx+c=0$: $x = \frac{-b \pm \sqrt{b^2-4ac}}{2a}$

	%\subsection*{Euler Identities}
	%	{\centering
	%	$Ae^{i\phi} = A(\cos\phi + i\sin\phi)$ \\
	%	$e^{z} = e^{x+iy} = e^{x}(\cos y + i \sin y)$  \\
	%	$\sin\phi = \frac{e^{i\phi}-e^{-i\phi}}{2i}$ \\
	%	$\cos\phi = \frac{e^{i\phi}+e^{-i\phi}}{2}$ \\
	%	}
		%\begin{eqnarray*}
		%	Ae^{i\phi} = A(\cos\phi + i\sin\phi)\\
		%	e^{z} = e^{x+iy} = e^{x}(\cos y + i \sin y)\\
		%	\sin\phi = \frac{e^{i\phi}-e^{-i\phi}}{2i}\\
		%	\cos\phi = \frac{e^{i\phi}+e^{-i\phi}}{2}
		%\end{eqnarray*}

	\subsection*{Trig}
	\textbf{Identities:}
	\begin{center}
		$\displaystyle\sin (A\pm B) =  \sin A\cos B\pm \cos A\sin B$ \\
		$\displaystyle\cos (A\pm B) =  \cos A\cos B\mp \sin A\sin B$ \\
		\vspace{4pt}
		$\displaystyle\sin A\sin B =
			\frac{1}{2}\left[\cos(A-B)-\cos(A+B)\right]$\\
		$\displaystyle\cos A\cos B =
			\frac{1}{2}\left[\cos(A-B)+\cos(A+B)\right]$\\
		$\displaystyle\sin A\cos B =
			\frac{1}{2}\left[\sin(A+B)+\sin(A-B)\right]$\\
		$\displaystyle\cos A\sin B =
			\frac{1}{2}\left[\sin(A+B)-\sin(A-B)\right]$\\
		\vspace{4pt}
		$\displaystyle\sin^{2}A =  \frac{1-\cos 2A}{2}$ \\
		$\displaystyle\cos^{2}A =  \frac{1+\cos 2A}{2}$ \\
		\vspace{4pt}
		$\displaystyle\sin A + \sin B =  2\sin\left(\frac{A+B}{2}\right)
			\cos\left(\frac{A-B}{2}\right)$ \\
		$\displaystyle\sin A - \sin B =  2\cos\left(\frac{A+B}{2}\right)
			\sin\left(\frac{A-B}{2}\right)$ \\
		$\displaystyle\cos A + \cos B =  2\cos\left(\frac{A+B}{2}\right)
			\cos\left(\frac{A-B}{2}\right)$ \\
		$\displaystyle\cos A - \cos B =  -2\sin\left(\frac{A+B}{2}\right)
			\sin\left(\frac{A-B}{2}\right)$
	\end{center}
		\textbf{Law of Sines:} sides: $A$, $B$, and $C$; angles opposite:
		$\alpha$, $\beta$, and $\gamma$
	\begin{center}
		$\frac{A}{\sin \alpha} = \frac{B}{\sin \beta} = \frac{C}{\sin
			\gamma}$
	\end{center}
		\textbf{Law of Cosines:} sides: $A$, $B$, and $C$; angle opposite of
		$C$ = $\gamma$
	\begin{center}
		$C^2 = A^2 + B^2 - 2AB \cos \gamma$
	\end{center}
		\textbf{Circular Arclength:} $s = r\theta$; $\theta$ in rad

	%\subsection*{Fundamental theorem of calculus}
	%	\textbf{First part}: Define $F(x) = \int_a^x f(t) dt$ in interval
	%	$[a,b]$ with $f$ continuous and real-valued in $[a,b]$.  Then, $F$ is
	%	continuous on $[a,b]$, differentiable on $(a,b)$, and $F'(x) = f(x) \,
	%	\forall x \in [a,b]$\\
	%	\vspace{4pt}
	%	\textbf{Corollary}: If $f$ is a real-valued continuous function on
	%	$[a,b]$, and $g$ is an antiderivative of $f$ in $[a,b]$, then $\int_a^b
	%	f(x)dx = g(b)-g(a)$.\\
	%	\vspace{4pt}
	%	\textbf{Second part (stronger than corollary)}: Let $f$ be a real-valued
	%	function defined on $[a,b]$ with an antiderivative $g$ on $[a,b]$
	%	(i.e., $f(x)=g'(x) \,\, \forall x \in [a,b]$).  If $f$ is integrable on
	%	$[a,b]$ then $\int_a^bf(x)dx = g(b)-g(a)$.  Note that here $f$ needn't
	%	be continuous.

	\subsection*{Vectors}
	\textbf{Divergence theorem (Gauss' thm):}
		$ \int_{V}(\nabla\cdot \mathbf{F}) \, d_{V} =
			\oiint_S \mathbf{F} \cdot \mathbf{n} \, dS $\\
			%\oint_S \mathbf{F} \cdot \mathbf{n} \, dS $\\
	\textbf{Stokes theorem:}
	$ \int_S \nabla \times \mathbf{F} \cdot dS = \oint_C \mathbf{F}\cdot d\mathbf{r} $\\
	\textbf{Dot product:} $ \mathbf{A} \cdot \mathbf{B} = |A||B|\cos\theta\mathbf{\hat{n}} = A_xB_x+A_yB_y+A_zB_z$ \\
	\textbf{Cross product:}$ \mathbf{A} \times \mathbf{B} = |A||B|\sin\theta \mathbf{\hat{n}}  $\\


	\subsection*{Coordinate systems \& conversions}
	\begin{center}\textbf{COORDINATE CONVERSION}\\
	\begin{tabular}{ c c c }
		\texttt{cartesian to\dots} & \texttt{cylindrical to\dots} &
			\texttt{spherical to\dots} \\
		\vspace{.5pt}\\
		\hline
		\vspace{.5pt}\\
		\texttt{cylindrical}    & \texttt{cartesian} & \texttt{cartesian} \\
		\vspace{.5pt}\\
		\hline
		\vspace{.5pt}\\
		$\rho=\sqrt{x^2+y^2}$ & $x=\rho\cos\phi$ & $x=r\sin\theta\cos\phi$ \\
		$\phi=\arctan(y/x)$    & $y=\rho\sin\phi$ & $y=r\sin\theta\sin\phi$ \\
		$z = z$               & $z=z$            & $z=r\cos\theta$\vspace{2pt}\\
		
		\vspace{.5pt}\\
		\hline
		\vspace{.5pt}\\
		\texttt{spherical}      & \texttt{spherical} & \texttt{cylindrical} \\
		\vspace{.5pt}\\
		\hline
		\vspace{.5pt}\\
		
		$r=\sqrt{x^2+y^2+z^2}$& $r=\sqrt{\rho^2+z^2}$&$\rho=r\sin\theta$ \\
		$\theta=\arccos(z/r)$  & $\theta=\arctan(\rho/z)$&$\phi=r\sin\theta$ \\
		$\phi=\arctan(y/x)$    & $\phi=\phi$      & $z=r\cos\theta$ \\
	\end{tabular} \vspace{4pt}\\

	\textbf{UNIT VECTOR CONVERSION}
	\begin{tabular}{ c c c }
		\texttt{cartesian to\dots} & \texttt{cylindrical to\dots} &
			\texttt{spherical to\dots} \\

		\vspace{.5pt}\\
		\hline
		\vspace{.5pt}\\
		\texttt{cylindrical}    & \texttt{cartesian} & \texttt{cartesian} \\
		\vspace{.5pt}\\
		\hline
		\vspace{.5pt}\\

		$ \bm{\hat{\rho}} = \frac{x}{\rho}\bm{\hat{x}} +
				\frac{y}{\rho}\bm{\hat{y}} $ &
		$ \bm{\hat{x}} = cos\phi\bm{\hat{\rho}} - sin\phi\bm{\hat{\phi}} $ &
		\vspace{-8pt}
		\parbox[t]{.9in}{\vspace{-15pt}
			\begin{equation*}\begin{split}
				\bm{\hat{x}} =&
					\sin\theta\cos\phi\bm{\hat{r}} + \\
					&{\cos\theta\cos\phi\bm{\hat{\theta}} - 
					\sin\phi\bm{\hat{\phi}}}
			\end{split}\end{equation*} } \\

		$ \bm{\hat{\phi}} = -\frac{y}{\rho}\bm{\hat{x}} + 
			\frac{x}{\rho}\bm{\hat{y}} $ &
		$ \bm{\hat{x}}=cos\phi\bm{\hat{\rho}}-sin\phi\bm{\hat{\phi}} $ &
		\vspace{-8pt}
		\parbox[t]{.9in}{\vspace{-15pt}
			\begin{equation*}\begin{split}
				\bm{\hat{y}} =& \sin\theta\sin\phi\bm{\hat{r}} + \\
					&\cos\theta\sin\phi\bm{\hat{\theta}} +
					\cos\phi\bm{\hat{\phi}}
				\end{split}\end{equation*} } \\

		$ \bm{\hat{z}} = \bm{\hat{z}} $ &
		$ \bm{\hat{z}} = \bm{\hat{z}} $ &
		$ \bm{\hat{z}} = \cos\theta\bm{\hat{r}}-\sin\theta\bm{\hat{\theta}} $\\

		\vspace{.5pt}\\
		\hline
		\vspace{.5pt}\\
		\texttt{spherical}      & \texttt{spherical} & \texttt{cylindrical} \\
		\vspace{.5pt}\\
		\hline
		\vspace{.5pt}\\

		$ \bm{\hat{r}} = \frac{x\bm{\hat{x}} + y\bm{\hat{y}} +
			z\bm{\hat{z}}}{r} $ &
		$ \bm{\hat{r}} = \frac{\rho}{r}\bm{\hat{\rho}} + 
			\frac{z}{r}\bm{\hat{z}} $ &
		$ \bm{\hat{\rho}} = \sin\theta\bm{\hat{r}} +
			\cos\theta\bm{\hat{\theta}} $ \\

		$ \bm{\hat{\theta}} = \frac{xz\bm{\hat{x}} + yz\bm{\hat{y}} -
			\rho^2\bm{\hat{z}}} {r\rho} $ &
		$ \bm{\hat{\theta}} = \frac{z}{r}\bm{\hat{\rho}} -
			\frac{\rho}{r}\bm{\hat{z}} $ &
		$ \bm{\hat{\phi}} = \bm{\hat{\phi}} $ \\

		$ \bm{\hat{\phi}} = \frac{-y\bm{\hat{x}} + x\bm{\hat{y}}}{\rho} $ &
		$ \bm{\hat{\phi}} = \bm{\hat{\phi}} $ &
		$ \bm{\hat{z}} = \cos\theta\bm{\hat{r}} - \sin\theta\bm{\hat{\theta}}$\\
		
	\end{tabular}\\
	\end{center}
	
	\begin{center}
		\textbf{DIFFERENTIAL ELEMENTS}\\
		\begin{tabular}{ c  c  c }
			\texttt{cartesian} & \texttt{cylindrical} & \texttt{spherical} \\

			\vspace{-4pt}
			\parbox[t]{.9in}{\vspace{-10pt}
				\begin{equation*}\begin{split}
					d\bm{\mathbf{l}} =& dx\bm{\hat{x}} + dy\bm{\hat{y}} +
						dz\bm{\hat{z}}
				\end{split}\end{equation*} } &

			\parbox[t]{.9in}{\vspace{-10pt}
				\begin{equation*}\begin{split}
					d\bm{\mathbf{l}} =& d\rho\bm{\hat{\rho}} +
						\rho d\phi\bm{\hat{\phi}} + dz\bm{\hat{z}}
				\end{split}\end{equation*} } &

			\parbox[t]{.9in}{\vspace{-10pt}
				\begin{equation*}\begin{split}
					d\bm{\mathbf{l}} = dr\bm{\hat{r}} + rd\theta\bm{\hat{\theta}}
						+ & \\
						r\sin\theta d\phi\bm{\hat{\phi}}
				\end{split}\end{equation*} } \\

			\vspace{-4pt}
			\parbox[t]{.9in}{\vspace{-15pt}
				\begin{equation*}\begin{split}
					d\bm{\mathbf{A}} =& dy\, dx\, \bm{\hat{x}} + \\
						&dx\, dz\, \bm{\hat{y}} + \\
						&dx\, dy\, \bm{\hat{z}}
				\end{split}\end{equation*} } &

			\parbox[t]{.9in}{\vspace{-15pt}
				\begin{equation*}\begin{split}
					d\bm{\mathbf{A}} =& \rho d\phi\, dz\, \bm{\hat{\rho}} + \\
						&d\rho\, dz\, \bm{\hat{\phi}} + \\
						&\rho\, d\rho\, d\phi\, \bm{\hat{z}}
				\end{split}\end{equation*} } &

			\parbox[t]{.9in}{\vspace{-15pt}
				\begin{equation*}\begin{split}
					d\bm{\mathbf{A}} =& r^2\sin\theta \, d\theta \, d\phi \,
							\bm{\hat{r}} + \\
						&r\sin\theta\, dr \, d\phi \, \bm{\hat{\theta}} + \\
						& r \, dr \, d\theta \, \bm{\hat{\phi}}
				\end{split}\end{equation*} } \\

			\vspace{-8pt}
			\parbox[t]{.9in}{\vspace{-15pt}
				\begin{equation*}\begin{split}
					dV =& dx \, dy \, dz
				\end{split}\end{equation*} } &

			\parbox[t]{.9in}{\vspace{-15pt}
				\begin{equation*}\begin{split}
					dV =& \rho \, d\rho \, d\phi \, dz
				\end{split}\end{equation*} } &

			\parbox[t]{.9in}{\vspace{-15pt}
				\begin{equation*}\begin{split}
					dV =& r^2 \, \sin\theta \, dr \, d\theta \, d\phi
				\end{split}\end{equation*} } \\
		
		\end{tabular}\\
	\end{center}

	\begin{center}
		\textbf{POS, VEL, ACCEL}\\
	\end{center}
	\textbf{polar:}\\
	$\bm{\mathbf{r}} = \rho\bm{\mathbf{e}}_\rho$\\
	$\bm{\mathbf{v}} = \dot{\rho}\bm{\mathbf{e}}_\rho + \rho\dot\theta\bm{\mathbf{e}}_\theta$\\
	$\bm{\mathbf{a}} = \left(\ddot{\rho}-\rho\dot\theta^2\right)\bm{\mathbf{e}_\rho} + 
	\left( \rho\ddot\theta+2\dot{\rho}\dot\theta\right)\bm{\mathbf{e}_\theta}$\\
	\textbf{cylindrical:}\\
	$\bm{\mathbf{r}} = \rho\bm{\mathbf{e}}_\rho + z\bm{\mathbf{e}}_z$\\
	$\bm{\mathbf{v}} = \dot{\rho}\bm{\mathbf{e}}_\rho + \rho\dot\theta\bm{\mathbf{e}}_\theta + \dot{z}\bm{\mathbf{e}_z}$\\
	$\bm{\mathbf{a}} = \left(\ddot{\rho}-\rho\dot\theta^2\right)\bm{\mathbf{e}_\rho} + 
	\left( \rho\ddot\theta+2\dot{\rho}\dot\theta\right)\bm{\mathbf{e}_\theta} +
	\ddot{z}\bm{\mathbf{e}_z}$

	\textbf{spherical:}\\
	$\bm{\mathbf{r}} = \rho \bm{\mathbf{e}_\rho}$\\
	$\bm{\mathbf v} = \dot \rho \bm{\mathbf{e}_\rho} + \rho\dot \theta \bm{\mathbf{e}_\theta} +
		\rho \dot\phi\sin\theta\bm{\mathbf{e}_\phi}$\\
	$\bm{\mathbf{a}} = \left( \ddot \rho -
		\rho\dot\theta^2-\rho\dot\phi^2\sin^2\theta\right)\bm{\mathbf{e}_\rho} +
		\left(\rho\ddot\theta+2\dot \rho \dot\theta -
		\rho\dot\phi^2\sin\theta\cos\theta\right)\bm{\mathbf{e}_\theta} + \left(
		\rho\ddot\phi\sin\theta+2\dot \rho\dot\phi\sin\theta +
		2\rho\dot\theta\dot\phi\cos\theta\right)\bm{\mathbf{e}_\phi}$
	\begin{center}\textbf{del, $\mathbf{\nabla}$, in CARTESIAN}\end{center}
		\begin{mydescription}
			\item[del operator:]
				$\mathbf{\nabla} =
				\mathbf{e}_x \frac{\partial}{\partial x} +
				\mathbf{e}_y\frac{\partial}{\partial y} + 
				\mathbf{e}_z\frac{\partial}{\partial z}$  
			\item[gradient:]
				$\mathbf{\nabla}\phi =
				grad\,\phi =
				\mathbf{e}_x\frac{\partial\phi}{\partial x} +
				\mathbf{e}_y\frac{\partial\phi}{\partial y} +
				\mathbf{e}_z\frac{\partial\phi}{\partial z}$  
			\item[directional derivative:]
				$\frac{d\phi}{ds} =
					\mathbf{\nabla}\phi \cdot \frac{\mathbf{A}}{|\mathbf{A}|}$  
			\item[divergence:]
				$\mathbf{\nabla}\cdot\mathbf{V} =
				div\, \mathbf{V} =
					\frac{\partial V_x}{\partial x} +
					\frac{\partial V_y}{\partial y} +
					\frac{\partial V_z}{\partial z}$  
			\item[curl:]
				$\mathbf{\nabla}\times \mathbf{V} =
				\mathbf{e}_x\left(\frac{\partial V_z}{\partial y} -
					\frac{\partial V_y}{\partial z}\right) +
					\mathbf{e}_y \left( \frac{\partial V_x}{\partial z} -
					\frac{\partial V_z}{\partial x}\right) +
					\mathbf{e}_z \left( \frac{\partial V_y}{\partial x} -
					\frac{\partial V_x}{\partial y} \right) $  
			\item[Laplacian:]
				$\Delta f = \mathbf{\nabla}^2\phi =
				\mathbf{\nabla} \cdot (\mathbf{\nabla}\phi) = div\, grad\, \phi =
					\frac{\partial^2 \phi}{\partial x^2} +
					\frac{\partial^2 \phi}{\partial y^2} +
					\frac{\partial^2 \phi}{\partial z^2}$ 
		\end{mydescription}

		\vspace{-8pt}

		\begin{center}\textbf{del, $\mathbf{\nabla}$, in CYLINDRICAL}\end{center}
		\begin{mydescription}
			\item[gradient:]
				$\mathbf{\nabla}f =
				grad\,f =
				\frac{\partial f}{\partial\rho}\mathbf{e}_\rho +
				\frac{1}{\rho}\frac{\partial f}{\partial\phi}\mathbf{e}_\phi +
				\frac{\partial f}{\partial z}\mathbf{e}_z$
			\item[divergence:]
				$\mathbf{\nabla}\cdot\mathbf{V} =
				div\, \mathbf{V} =
				\frac{1}{\rho}\frac{\partial(\rho\mathbf{V}_\rho)}{\partial\rho} +
				\frac{1}{\rho}\frac{\partial \mathbf{V}_\phi}{\partial\phi} +
				\frac{\partial \mathbf{V}_z}{\partial z}
				$
			\item[curl:]
				$\mathbf{\nabla}\times \mathbf{V} =
				\left({ \frac{1}{\rho}\frac{\partial V_z}{\partial\phi} -
					\frac{\partial V_\phi}{\partial z}}\right)\mathbf{e}_\rho+
				\left({ \frac{\partial V_\rho}{\partial z} -
					\frac{\partial  V_z}{\partial\rho}}\right)\mathbf{e}_\phi+
				\frac{1}{\rho}\left({
					\frac{\partial(\rho V_\phi)}{\partial\rho} -
					\frac{\partial V_\rho}{\partial\phi}}\right)\mathbf{e}_z
				$
			\item[Laplacian:]
				$\Delta f = \mathbf{\nabla}^2f =
				\mathbf{\nabla} \cdot (\mathbf{\nabla}f) =
				\frac{1}{\rho}\frac{\partial}{\partial\rho}\left({
				\rho\frac{\partial f}{\partial\rho}}\right) +
				\frac{1}{\rho^2}\frac{\partial^2 f}{\partial\phi^2} +
				\frac{\partial^2 f}{\partial z^2}
				$
		\end{mydescription}
		
		\vspace{-8pt}

		\begin{center}\textbf{del, $\mathbf{\nabla}$, in SPHERICAL}\end{center}
		\begin{mydescription}
			\item[gradient:]
				$\mathbf{\nabla}f = grad\,f =
				\frac{\partial f}{\partial r}\mathbf{e}_r +
				\frac{1}{r}\frac{\partial f}{\partial \theta}\mathbf{e}_\theta +
				\frac{1}{r\sin\theta}\frac{\partial f}{\partial\phi}\mathbf{e}_\phi
				$
			\item[divergence:]
				$\mathbf{\nabla}\cdot\mathbf{V} = div\, \mathbf{V} =
				\frac{1}{r^2}\frac{\partial(r^2V_r)}{\partial r} +
				\frac{1}{r\sin\theta}\frac{\partial}{\partial\theta}\left({
				V_\theta\sin\theta}\right) +
				\frac{1}{r\sin\theta}\frac{\partial V_\phi}{\partial\phi}
				$
			\item[curl:]
				$\mathbf{\nabla}\times \mathbf{V} =
				\frac{1}{r\sin\theta}\left({ \frac{\partial}{\partial\theta}(V_\phi\sin\theta)-\frac{\partial V_\theta}{\partial\phi} }\right)\mathbf{e}_r +
				\frac{1}{r}\left({\frac{1}{\sin\theta}\frac{\partial V_r}{\partial\phi} - \frac{\partial}{\partial r}(r V_\phi) }\right) \mathbf{e}_\theta +
				\frac{1}{r}\left({\frac{\partial}{\partial r}(r V_\theta) -
				\frac{\partial V_r}{\partial\theta} }\right) \mathbf{e}_\phi
				$
			\item[Laplacian:]
				$\Delta f = \mathbf{\nabla}^2f =
				\mathbf{\nabla} \cdot (\mathbf{\nabla}f) =
				\frac{1}{r^2}\frac{\partial}{\partial r}\left( 
				r^2\frac{\partial f}{\partial r}\right) +
				\frac{1}{r^2\sin\theta}\frac{\partial}{\partial\theta}\left(
				sin\theta\frac{\partial f}{\partial\theta}\right) +
				\frac{1}{r^2\sin^2\theta}\frac{\partial^2f}{\partial\phi^2}
				$
		\end{mydescription}
		

	%\subsection*{Fourier series}
	%	\begin{mydescription}
	%		\item[real-valued functions, period of $2l$:]
	%		$ f(x) = \frac{a_0}{2} + \sum_{n=1}^\infty \left(
	%				a_n\cos \frac{n\pi x}{l} +
	%				b_n \sin \frac{n \pi x}{l}\right) $
	%		\begin{eqnarray*}
	%			a_0 &=& \frac{1}{l}\int_{-l}^{l} \! f(x) \, dx\\
	%			a_n &=& \frac{1}{l}\int_{-l}^{l} \!
	%				f(x) cos \frac{n\pi x}{l} \, dx, \,\, n \geq 1\\
	%			b_n &=& \frac{1}{l}\int_{-l}^{l} \!
	%				f(x)sin \frac{n\pi x}{l} \, dx, \,\, n \geq 1\\
	%		\end{eqnarray*}
	%		\item[complex-valued functions,]
	%			period of $2l$: \\
	%			$ f(x) = \sum_{n=-\infty}^{\infty} c_n e^{in\pi x/l} $
	%			$$
	%				c_n = \frac{1}{2l}\int_{-l}^{l} \! f(x)e^{-in\pi x/l} \, dx,
	%					\,\, n \in \mathbb{Z}
	%			$$
	%		\item[convergence]
	%			(Dirichlet): 
	%			If $f(x)$ is periodic of period $2 l$, and if between $-l$ and
	%			$l$ it is single-valued, has a finite number of max. and min.
	%			values, and a finite number of discont., and if
	%			$\int_{-l}^{l} \, |f(x)| \, dx$ is finite, Fourier series
	%			converges to $f(x)$ at all points where $f(x)$ is continuous.
	%			At discontinuities, series converges to midpoint of the jump.\\
	%	\end{mydescription}
   
	\subsection*{Taylor series}
		Taylor series of $f(x)$ about $x=a$:\\
		$$
			f(x) = f(a) + (x-a)f'(a) + \frac{1}{2!}(x-a)^2f''(a) + \cdots +
				\frac{1}{n!}(x-a)^nf^{(n)}(a) + \cdots
		$$
	
	%\subsection*{Green's method}
	%	$$
	%		x(t) = \int_{-\infty}^{t}F(t')G(t,t')dt'
	%	$$
	
	\subsection*{Ordinary differential equations}
		\subsubsection*{Separable 1\textsuperscript{st}-order}
			Equation can be written as $f(y)dy = f(x)dx$, such as
			$\frac{dy}{dx} = N(1-y)$.  Evaluate integrals directly.

		\subsubsection*{Linear 1\textsuperscript{st}-order}%: \\
			Write the equation in the form $ y' + P(x)y = Q(x) $ and then define
			$ I = \int P(x)dx $ and find $y$ by solving $ ye^{I} = \int Q(x)e^{I}dx+c $

		%\subsubsection*{Homogeneous \textital{Functions in}:
			%1\textsuperscript{st}-Order}:(Note that this is NOT the same
			%as a %Homogeneous) \underline{Homogeneous \textital{functions}
			%(NOT diff. eq's}: Write the equation in the form

		\subsubsection*{Linear 2\textsuperscript{nd}-order homogeneous with
				constant coefficients}
			Equations of the form $a_2 \frac{d^2y}{dx^2} + a_1 \frac{dy}{dx} + a_0y = 0$ \\
			%Write the characteristic polynomial $a_2D^2y + a_1 Dy + a_0y = 0$
			%and factor into $(D-a)(D-b)y = 0$.  In general, this can be solved
			%by letting $u=(D-a)y$, solving the 1\textsuperscript{st}-order diff
			%eq $(D-b)u=0$ for $u(x)$, substituting this solution into the
			%equation $(D-a)y=u(x)$, and finally solving \emph{this} linear
			%1\textsuperscript{st}-order ODE.  In fact, this method can be
			%generalized to higher-order linear diff eq's.  However, there are
			%pre-determined solution forms based upon the relationships between
			$a$ and $b$:\\
			$$
				a, b \in \Re, a \neq b \Rightarrow
					y=c_1e^{ax}+c_2e^{bx}
			$$
			$$
				a, b \in \Re, a = b \Rightarrow y=(Ax+B)e^{ax}
			$$
			For $a, b \in \Im, a = b^\ast = \alpha \pm i\beta,$ any of the
			following forms are solutions\\
			\begin{gather*}
				y=Ae^{\alpha + i\beta x} + Be^{\alpha - i\beta x}\\
				y=e^{\alpha x}\left(Ae^{i\beta x} + Be^{-i\beta x}\right)\\
				y=e^{\alpha x}\left(c_1\sin\beta x + c_2\cos\beta x\right)\\
				y=ce^{\alpha x}\sin\left( \beta x + \gamma \right)\\
				y=ce^{\alpha x}\cos\left( \beta x + \delta \right)
			\end{gather*}

		\subsubsection*{Linear 2\textsuperscript{nd}-order inhomogeneous
				with constant coefficients}
			Equations of one of the forms\\
			$$
				a_2 \frac{d^2y}{dx^2} + a_1 \frac{dy}{dx} + a_0y = f(x)
			$$
			$$
				\frac{d^2y}{dx^2} + \frac{a_1}{a_2} \frac{dy}{dx} +
					\frac{a_0}{a_2}y = F(x)
			$$
			%can be solved, generally, as described for the homogeneous case,
			%but with $F(x)$ on the right-hand side when solving the first
			%1\textsuperscript{st}-order ODE, $(D-b)u=F(x)$.  (This gives both
			%the particular \emph{and} complementary solution.)  Otherwise, find
			%$y = y_c + y_p$ where $y_c$, the complementary solution, comes from
			%solving the homogeneous equation and $y_p$ is a particular solution
			%from a pre-computed form for specific $F(x)$:\\
			\begin{center}
			$(D-a)(D-b)y=F(x)=ke^{cx}$, particular solution $y_p$ is given by:\\
			\begin{tabular}{ l l }
				$y_p=Ce^{cx}$ & if $c$ is not equal to either $a$ or $b$; \\
				$y_p=Cxe^{cx}$ & if $c$ equals $a$ or $b$, $a\neq b$;\\
				$y_p=Cx^2e^{cx}$ & if $c=a=b$ \\
			\end{tabular}\\
			\emph{(For $F(x)=k\cos\alpha x$ or $F(x)=k\sin\alpha x$, solve the
			above with $F(x)=ke^{c=i\alpha x}$ and take the real or imag part,
			respectively. For $F(x)=const$, set $c=0$.)}
			\end{center}
			A more general form of this (called the \emph{method of
			undetermined coefficients}) follows:\\
			\begin{center}
				$(D-a)(D-b)y=F(x)=e^{cx}P_n(x)$; $P_n(x)$ is a polynomial of
				degree $n$:
				\[
				y_p = 
				\begin{cases}
					   e^{cx}Q_n(x) & \text{if $c \neq a$ and $c \neq b$}\\
					  xe^{cx}Q_n(x) & \text{if $c=a$ or $c=b$, $a\neq b$}\\
					x^2e^{cx}Q_n(x) & \text{if $c=a=b$}
				\end{cases}
				\]
			\end{center}

		\subsection*{Calculus of variations}
			$J = \int_{x_1}^{x_2} f\{y(x), y\prime (x); x\}$\\
			$\frac{\partial f}{\partial y}-\frac{d}{dx}\frac{\partial
			f}{\partial y\prime} = 0$\\
			$\frac{\partial f}{\partial x}-\frac{d}{dx}\left(f-y\prime
			\frac{\partial f}{\partial y\prime}\right) = 0$\\
			$f-y\prime \frac{\partial f}{\partial y\prime} =
				const$ for $\frac{\partial f}{\partial x} = 0$

\section*{Fundamental (and not so fundamental) Constants}
	\begin{myitemize}
		%\item[$c = $] 
		%  speed of light in vacuum $ = 3.00\times 10^8$ m$/$s
		%\item[$\mu_0 = $]
		%	mag const / perm of vacuum $ = 4 \pi \times 10^{-7}$
		%	N$\cdot$A$^{-2}$ or H$\>$m$^{-1}$
		%\item[$\varepsilon_0 = $]
		%	elec const / permit of vacuum $ = 8.854 \times 10^{-12}$
		%	F$\>$m$^{-1}$
		%\item[$Z_0 = $]
		%	char impedance of vacuum $ = 376.73$ $\Omega$

		%\vspace{4pt}

		%\item[$e = $]
		%  charge of electron  $ = 1.602\times10^{-19}$ C
		%\item[$m_e = $]
		%  mass of electron $ = 9.11\times10^{-31}$ kg $ = 0.511$ MeV/$c^2$
		%\item[$m_n = $]
		%  mass of neutron or proton $ = 1.67\times10^{-27}$ kg
		%  $ = 938$ MeV/$c^2$

		%\vspace{4pt}

		\item[$G = $]
		  gravit. constant $ = 6.674\times10^{-11}$ N$\>$m$^2$kg$^{-2}$
		\item[$g = $]
		  gravit. accel on Earth surface $ = 9.8 \>$m$/$s$^{2}$
		\item[Note:] $GM_e=gR_e^2$

		\vspace{4pt}

		\item[$R_S = $]
		  mean radius of Sun $ = 696\times10^{6}$ m
		\item[$R_E = $]
		  mean radius of Earth $ = 6.371\times10^{6}$ m
		\item[$R_M = $]
		  mean radius of Moon $ = 1.737\times10^{6}$ m

		\vspace{4pt}

	  	\item[$R_{S_{E}} = $]
		  mean distance, Earth to Sun $ = 149.6\times10^{9}$ m
		\item[$D = $]
		  mean distance, Earth to Moon $ = 384.4\times10^{6}$ m

		\vspace{4pt}

		\item[$M_S = $]
		  mass of Sun $ = 1.99\times10^{30}$ kg
		\item[$M_E = $]
		  mass of Earth $ = 5.98\times10^{24}$ kg
		\item[$M_m = $]
		  mass of Moon $ = 7.35\times10^{22}$ kg

		%\vspace{4pt}

		%\item[$k_B = $]
		%	Boltzmann's constant $ = 1.38 \times 10^{-23} \>$J/K
		%\item[$N_A = $]
		%  Avogadro's number $ = 6.02214179 \times 10^{23} \>$mol$^{-1}$

	\end{myitemize}

\section*{Physics}

	\subsection*{Newton's laws}
		\begin{mydescription}
			\item[1\textsuperscript{st}:]
				Body remains at rest or in uniform motion unless acted upon by
				a force  \\
			\item[2\textsuperscript{nd}:]
				$\mathbf{F}_{tot} = \frac{d\mathbf{p}}{dt} = m\mathbf{a}$	\\
			\item[3\textsuperscript{rd}:]
				$\mathbf{F}_{A \rightarrow B} = -\mathbf{F}_{B \rightarrow A}$\\
		\end{mydescription}
	
	\subsection*{Lagrangian dynamics}
			  \textbf{Hamilton's principle} --- Nature minimizes (makes
			  stationary) the action.
			  \textbf{Constrained} --- If a 3D system of $N$ particles has $n <
			  3N$ minimum generalized coordinates, the system is
			  \emph{constrained}. \\
			  \textbf{Natural} --- The coordinates $q_n$ are \emph{natural} if
			  the relationships of $r_\alpha$ (every particle's position) to
			  $q_n$ doesn't change with time.\\
			  \textbf{Ignorable} --- a coordinate $q_i$ is \emph{ignorable} if
			  the corresponding generalized momentum $p_i$ is constant.\\
			  \textbf{Lyupanov Stability} --- If $x_e$ is Lyapunov stable
			  and all solutions that start out near $x_e$ converge to $x_e$,
			  then $x_e$ is asymptotically stable
		\begin{mydescription}
			\item[Lagrangian:]
			  $\mathscr{L}=T-U$
			\item[Action:]
			  $S = \int_{t_1}^{t_2}\mathscr{L}(q_1,q_2,\,\dots\, ,q_N,\dot{q}_1,\dot{q}_2,\,\dots \,,\dot{q}_N,t)dt$
			\item[Euler-Lagrange equations:]$\frac{\partial \mathscr{L}}{\partial q_1}=\frac{d}{dt}\frac{\partial \mathscr{L}}{\partial \dot{q_1}},\,\dots$ etc.
			\item[Generalized forces:]$F_i=\frac{\partial \mathscr{L}}{\partial
			  q_i}$
			\item[Generalized momenta]$p_i = \frac{\partial \mathscr{L}}{\partial \dot{q_i}}$
		\end{mydescription}
	
	\subsection*{Orbits}
		\begin{mydescription}
		  \item[Definitions:] $M=m_1+m_2$; $\mu=\frac{m_1m_2}{m_1+m_2}$; e.g., $U(\rho)=\frac{-Gm_1m_2}{\rho}$
		  \item[Kinetic energy:] $T=\frac{1}{2}M\dot{r}^2+\frac{1}{2}\mu\dot{r}^2$
		  \item[Lagrangian:] $\mathscr{L}=\frac{1}{2}\mu\dot{\rho}^2+\frac{1}{2}\mu{\rho}^2\dot{\phi}^2-U(\rho)$
		  \item[Solution in $\phi$:] $\dot\phi=\frac{\ell}{\mu\rho^2}$ ($\ell$ const --- angular momentum)
		  \item[Solution in $\rho$:] $\mu\ddot{\rho}=-\frac{d}{d\rho}U(\rho) + \frac{\ell^2}{\mu\rho^3} = -\frac{d}{d\rho}\left[U(\rho)+\frac{\ell^2}{2\mu\rho^2}\right]$
		  \item[Effective potential:]$U_{eff}=U(\rho)+\frac{\ell^2}{2\mu\rho^2}$
		  \item[Note cons. of energy:] $\frac{d}{dt}\left( \frac{1}{2}\mu\dot\rho^2 \right) = -\frac{d}{dt}U_{eff}(\rho)$; $E=\frac{1}{2}\mu\dot\rho^2+U_{eff}(\rho)$
		  \item[Use:]$u=1/r$ and $\frac{d}{dt}=\frac{d\phi}{dt}\frac{d}{d\phi}=\dot\phi\frac{d}{d\phi}=\frac{\ell}{\mu\rho^2}\frac{d}{d\phi}=\frac{\ell u^2}{\mu}\frac{d}{d\phi}$
		  \item[$u$-equation:]$u''(\phi)=-u(\phi)-\frac{\mu}{\ell^2u(\phi)^2F(u)}$
		  \item[Use:]$\gamma=Gm_1m_2$ and $F(u)=-\gamma u^2$; then $U''(\phi)=-u(\phi)+\gamma\mu/\ell^2$; use $w(\phi)=u(\phi)-\gamma\mu/\ell^2$, so $W(\phi)=A\cos(\phi-\delta)$ ergo $u(\phi)=\frac{\gamma\mu}{\ell^2}+A\cos\phi$
		  \item[Radial eqn:] $r(\phi)=\frac{r_c}{1+\varepsilon\cos\phi}$
		  \item[Cartesian:] $$\left( \frac{x+\frac{r_c\varepsilon}{1-\varepsilon^2}}{\frac{r_c}{1-\varepsilon^2}} \right)^2+\left( \frac{y}{\frac{r_c}{\sqrt{1-\varepsilon^2}}} \right)^2$$
		  \item[Eccentricity:] $\varepsilon=A\cdot r_c$ ($A$ some constant)
		  \item[Circular orbit:] $r_c=\ell^2/\gamma\mu$
		  \item[Min radius:] $r_{min}=\frac{r_c}{1+\varepsilon}=\frac{\ell^2}{\gamma\mu(1+\varepsilon)}$ (at $\phi=0$; \textbf{perihelion}); $\ell=\mu rv_{tan}$
		  \item[Max radius:] $r_{max}=\frac{r_c}{1-\varepsilon}$ (at $\phi=\pi$; \textbf{aphelion})
		  \item[Radial velocity:]$v_r = \sqrt{\frac{\mu}{r_c}}\cdot\varepsilon\cdot\sin\phi$
		  \item[Tangential velocity:]$v_t = \sqrt{\frac{\mu}{r_c}}\cdot\left(1+\varepsilon\cdot\cos\phi\right)$
		  \item[Ellipse params:] $a=\frac{r_c}{1-\varepsilon^2}$; $b=\frac{r_c}{\sqrt{1-\varepsilon^2}}$; $d=a\varepsilon$; $\varepsilon=\sqrt{1-(b/a)^2}$
		  \item[Orbital period:] $\tau=2\pi\cdot a\cdot\sqrt{\frac{a}{\mu}}$
		  \item[Energy:] $E=\frac{\gamma^2\mu}{2\ell^2}(\varepsilon^2-1)$
		  \item[Kepler's $1^{st}$ law:] Orbits are ellipses w/ sun at a focus (approx. true)
		  \item[Kepler's $2^{nd}$ law:] Line from Sun to planet, equal areas in equal times\\
			$dA=\frac{1}{2}r^2d\phi$; $\frac{dA}{dt}=\frac{1}{2}\frac{\ell}{\mu}$, inep. of time
		  \item[Kepler's $3^{rd}$ law:] $\tau=\frac{A}{dA/dt}=\frac{2\pi ab\mu}{\ell} \Rightarrow \tau^2=4\pi^2\frac{a^3r_c\mu^2}{\ell^2}=4\pi^2\frac{a^3\mu}{\gamma}\approx \frac{4\pi^2}{GM_s}a^3$
		\end{mydescription}

	\subsection*{Cartesian system}
		\begin{mydescription}
			\item[inertia:] $I = \int_V r^2 dM$, $dM = \rho(x,y,z)dx dy dz$
			\item[momentum:] $\mathbf{p}\equiv m\mathbf{v}$
			\item[kinetic energy:] $T = \frac{1}{2}mv^2$
		\end{mydescription}
	
	\subsection*{Rotating system}
		\begin{mydescription}
			\item[moment of inertia:]
				$I = \int r^2dm$ or, for a point mass, $I = r^2m$, where $r$ is
				the perp. distance to axis of rotation  \\
			\item[parallel axis theorem:]
				$I_z=I_{cm}+md^2$; $I_{cm}$: inertia about center of mass, $m$:
				mass, $d$: distance between axes  \\
			\item[angular momentum:]
				$\mathbf{L}\equiv \mathbf{r} \times \mathbf{p} = I\mathbf{\Omega}$;
					$\mathbf{r}$: position vec, $\mathbf{p}$: linear momentum   \\
			\item[torque:]
				$\mathbf{N}\equiv \mathbf{r}\times \mathbf{F} = \dot{\mathbf{L}} =
				\mathbf{r}\times\dot{\mathbf{p}}$  \\
			\item[work:]
				$W = N \theta$, $\theta$ in rad \\
			\item[angle:]
				$\mathbf{\theta}$  \\
			\item[angular velocity:]
				$\mathbf{\Omega} =
				\dot{\mathbf{\theta}} =
					\frac{\mathbf{r}\times \mathbf{v}}{|\mathbf{r}|^2}$\\
			\item[linear velocity:]
				$\mathbf{v}=\mathbf{\Omega}\times \mathbf{r}$ \\
			\item[angular acceleration:]
				$\mathbf{\alpha} =
					\dot{\mathbf{\Omega}} = \ddot{\mathbf{\theta}} = \mathbf{a}_T/r$;
					$\mathbf{a}_T$ is tangential acceleration  \\
			\item[newton's 2\textsuperscript{nd}-law:]
					$\mathbf{N} = I\mathbf{\alpha}$
			\item[time deriv, unit vec in rotating frame:]
			  $\frac{d\mathbf{e}}{dt}=\mathbf{\Omega}\times\mathbf{e}$ ($\mathbf{e}$ fixed in
			  body)
			\item[time deriv, vec in rotating frame:]$\left( \frac{d\mathbf{r}}{dt}
			  \right)_{S_0} = \left( \frac{d\mathbf{r}}{dt} \right)_{S} +
			  \mathbf{\Omega} \times \mathbf{r}\,$ ($S_0$: inert, $S$: rot)
			\item[Newton's $2^{nd}$ in rotating frame:] $m\ddot{\mathbf{r}} =
			  \mathbf{F}+2m\dot{\mathbf{r}}\times\mathbf{\Omega}+m\left(
			  \mathbf{\Omega\times\mathbf{r}} \right)\times\mathbf{\Omega}$
		\end{mydescription}
	
	\subsection*{Conservative Force}
	Conditions, given that $\mathbf{F}$ has continuous first partial derivatives
	in a simply connected region\dots
		\begin{mydescription}
			\item[No curl anywhere:]
				$\mathbf{\nabla}\times\mathbf{F}=0$
			\item[Equal work regardless of path]: \\
				$W_C = \int_{C}\mathbf{F} \cdot d\mathbf{s} = constant \,\forall\,C$\\
				$W_C = \oint_{C}\mathbf{F} \cdot d\mathbf{s} = 0 \,\forall\,C$
			\item[$\mathbf{F} \cdot d\mathbf{r}$ is exact differential]
			\item[$\mathbf{F} = \nabla W$, $W$ single-valued]
			\item[Allows definition of potential:]
			  $\mathbf{F} = - \nabla \mathbf{U}$
		\end{mydescription}
	\subsection*{Specific Forces}
		\begin{mydescription}
			\item[gravity]: \\
				point mass or sph.-symm mass: $\mathbf{F} =
					-G \, {\frac{ M \, m}{r^2}}\mathbf{e}_r \approx -mg$
				on earth \\
				generally: $\mathbf{F} = -Gm \int_V
					\frac{\rho(\mathbf{r}')\mathbf{e}_r}{r^2}dv'$ \\
				grav field vector: $\mathbf{g} \equiv - \mathbf{\nabla} \Phi
					= \mathbf{F}/m$ \\
				grav potential, point mass: $\Phi = -G\frac{M}{r}$\\
				grav potential, mass distr: $\Phi = -G\int_V
					\frac{\rho(\mathbf{r}')}{r}dv'$\\
				potential energy: $U = m \Phi$\\
				Gauss' law for grav, int:
					$\oint_S \mathbf{g}\cdot \, d\mathbf{A} = -4\pi GM$\\
				Gauss' law for grav, dif: $\nabla\cdot\mathbf{g}=-4\pi G\rho$\\
				Poisson's equation: $\nabla^2\phi = 4\pi G\rho$, for rad-sym
				system, this is $\frac{1}{r^2}\frac{\partial}{\partial
				r}\left(r^2\frac{\partial\phi}{\partial r}\right) = 4\pi G
				\rho(r)$ and $\mathbf{g}(r)=-\mathbf{e}_r
				\frac{\partial\phi}{\partial r}$

			%\vspace{4pt}

		\item[tidal] (due to Moon's gravity) \\
				$\mathbf{e}_R$ points from Moon's center to test mass on Earth \\
				$\mathbf{e}_D$ is from center of Moon to center of Earth \\
				$(x,y)$ ECEF coord of test mass\\
				$\mathbf{F}_T=-GmM_m \left( \frac{\mathbf{e}_R}{R^2} -
				\frac{\mathbf{e}_D}{D^2}\right)$\\
				$F_{T_x} \approx 2GmM_mx/D^3$\\
				$F_{T_y} \approx -GmM_my/D^3$

		\item[spring] (simple, linear) \\
			$F = -kx$ ($x$: displ from eq lib, $k$: spring const)  \\
			$U = \frac{1}{2}kx^2$\\

		\item[inertial force, linear accel:] $\mathbf{F_{inert}}=-m\mathbf{A}$
			($\mathbf{A}$: frame's accel w.r.t. inertial frame)

		\item[centrifugal] (inertial force) \\
		  $\mathbf{F_{centr}}=m\left( \mathbf{\Omega}\times\mathbf{r}
		  \right)\times\mathbf{\Omega}$ (generally)\\
		  $\mathbf{F_{centr}} = \frac{mv^2}{r}\mathbf{e}_r =
		   mr\Omega^2\mathbf{e}_r$ (for circular motion)\\
		  $U_{centr}(r) = \frac{\ell^2}{2mr^2}$ ($\ell$: angular momentum)\\
		  Free-fall accel (e.g., on Earth): $\mathbf{g}=\mathbf{g}_0+\left(
		  \mathbf{\Omega\times\mathbf{R}} \right)\times\mathbf{\Omega}$

		\item[coriolis] (inertial force)\\
		  $\mathbf{F_{cor}}=2m\dot{\mathbf{r}}\times \mathbf{\Omega}$

		%\item[friction] (pseudo-force) \\
		%	$\mathbf{F}_f = \mu F_N$ ($\mu$: static ($\mu_s$) or
		%	kinetic ($\mu_k$), $F_N$: normal force)\\
		%	Angle of friction (obj starts to move): $tan \theta = \mu_s$\\
		%	Energy converted to heat: $E_{th} = \mu_k \int F_n(x)dx$

		%\item[general retarding] --- \\
		%	$\mathbf{F} = -bm\dot{\mathbf{x}}^n$ ($b$: damping const,
		%	$m$: mass, $n$: power of velocity dep., just 1 in simple cases)  \\

		%\item[air resistance / drag] --- \\
		%	$W=\frac{1}{2}c_W\rho Av^2$,
		%	$c_W$: dimensionless drag coeff, $\rho$: air density,
		%	$A$: cross-sectional area perp. to velocity ($v$) \\

		\item[buoyant] --- \\
			$F = \rho_{fluid}Vg$, dir. opposite to grav.-induced pressure grad.
			in fluid; $\rho_{fluid}$: density, $V$: submerged volume,
			$g$: grav. \\

		\item[lorentz] --- \\
			$\mathbf{F} = q\mathbf{v}\times\mathbf{B}$,
			$q$ is charge of particle, $\mathbf{v}$ its velocity,
			$\mathbf{B}$ is mag. field strength  \\

		\item[electrostatic] --- \\
			$\mathbf{F} = q\mathbf{E}$, $q$ is charge, $\mathbf{E}$ electric field
		\end{mydescription}
	
	\subsection*{Energy}
		\begin{mydescription}
			\item[potential energy:]
				$\displaystyle\int_1^2\mathbf{F}\cdot d\mathbf{r} \equiv U_1 - U_2$\\
				(work, done by force $\mathbf{F}$, req'd to move particle from
				point 1 to point 2 with no change in kinetic energy); potential
				energy is the capacity to do work.  \\
			\item[force due to the potential $U$:]
				$\mathbf{F} = -\mathbf{\nabla}U$  \\
			\item[kinetic energy:]
				$T \equiv \frac{1}{2}mv^2$  \\
			\item[total energy:]
				$E \equiv T + U$  \\
			\item[1D solution given $E$ and $U(x)$,]
				for conservative force only:\\
				$$
					t-t_0 = \int_{x_0}^x\frac{\pm dx}
					{\sqrt{\frac{2}{m}\left[E-U(x)\right]}}
				$$
		\end{mydescription}

		%\item[potential energy:]
		%	$\int_1^2\vec{f}\cdot d\vec{r} \equiv U_1 - U_2$ (work, done by
		%	force $\vec{F}$, req'd to move particle from point 1 to point 2
		%	with no change in kinetic energy); potential energy is the capacity
		%	to do work.  \\
		%\item[force due to the potential $U$:]
		%	$\vec{F} = -\vec{\nabla}U$  \\
		%\item[kinetic energy:]
		%	$T \equiv \frac{1}{2}mv^2$  \\
		%\item[total energy:]
		%	$E \equiv T + U$  \\
		%\item[1D solution given $E$ and $U(x)$], for conservative force
		%only:\\
		%	$$
		%		t-t_0 = \int_{x_0}^x\frac{\pm dx}
		%		{\sqrt{\frac{2}{m}\left[E-U(x)\right]}}
		%	$$
	
	\subsection*{Conservation theorems}
		\begin{mydescription}
			\item[linear momentum:]
				$\frac{d}{dt}\left(p_1 + p_2\right) = 0$ (or $p_1+p_2$ is const)
				if no external forces act upon system  \\
			\item[angular momentum:]
				$\dot{\mathbf{L}}=\mathbf{r}\times\dot{\mathbf{p}}=0$ (or $\mathbf{L}$ is
				const) if no external torque acts upon system \\
			\item[energy:]
				$\mathbf{F}+\nabla U = 0$; $\frac{dE}{dt} = 0$ if the force field
				represented by $\mathbf{F}$ is conservative
		\end{mydescription}
	
	\subsection*{Harmonic oscillation}
	\subsubsection*{Simple harmonic oscillator}
		{\centering
		$m\ddot{x} = -kx$  \\
		$\omega_0^2 \equiv k/m$  \\
		$\ddot{x} + \omega_0^2x = 0$  \\
		$x(t) = A\sin (\omega_0 t - \delta)$  \\
		$E = T + U = \frac{1}{2}kA^2$ \\
	
		}
	
	\subsubsection*{Damped oscillator}
		\begin{mydescription}
			\item[equation of motion:]
				$m\ddot{x} + b\dot{x} + kx = 0$; $b$ is resisting force coeff,
				$k$ is restoring force coeff  \\
			\item[convenient substitutions:]
				$\beta \equiv \frac{b}{2m}$ (damping), and $\omega_0^2 \equiv
				k/m$ (natural ang. freq, undamped sys)  \\
			\item[new eqn of motion:]
				$\ddot{x} + 2\beta\dot{x} + \omega_0^2x = 0$  \\
			\item[general sol'n:]
				$$
					x(t) = e^{-\beta t}\left[ A_1exp\left(
					\sqrt{\beta^2-\omega_0^2}t\right) + A_2exp\left(
					-\sqrt{\beta^2-\omega_0^2}t\right) \right]
				$$
			\item[underdamping:]
				$\omega_0^2 > \beta^2$  \\
			\item[critical damping:]
				$\omega_0^2 = \beta^2$  \\
			\item[overdamping:]
				$\omega_0^2 < \beta^2$  \\
		\end{mydescription}
	
	\subsubsection*{Sinusoidally-driven damped oscillator}
		\begin{mydescription}
			\item[eqn of motion:]
				$m\ddot{x} + b\dot{x} + kx = F_0\cos\omega t$;
				$b$ is resisting force coeff, $k$ is restoring force coeff  \\
			\item[convenient substitutions:]
				$A=F_0/m$ (driving ampl), $\beta \equiv \frac{b}{2m}$
				(damping), and $\omega_0^2 \equiv k/m$ (natural ang. freq,
				undamped sys)  \\
			\item[new eqn of motion:]
				$\ddot{x} + 2\beta\dot{x} + \omega_0^2x = A\cos\omega t$  \\
			\item[complementary solution:]
				$$
					x_c(t) = e^{-\beta t}\left[ A_1exp\left(
					\sqrt{\beta^2-\omega_0^2}t\right) + A_2exp\left(
					-\sqrt{\beta^2-\omega_0^2}t\right) \right]
				$$
			\item[particular solution:]
				$$
					x_p(t)=\frac{A}{\sqrt{\left(
					\omega_0^2-\omega^2\right)^2+4\omega^2\beta^2}} cos(\omega
					t-\delta)
				$$
				$$
					\delta = \arctan\left( \frac{2\omega\beta}{\omega_0^2 -
					\omega^2} \right)
				$$
			\item[amplitude resonance frequency:]
				$\omega_R = \sqrt{\omega_0^2-2\beta^2} \,\, (\omega_r <
				\omega_1 < \omega_0)$ \\
			\item[kinetic energy resonance frequency:]
				$\omega_E = \omega_0$ \\
			\item[quality factor:]
				$Q \equiv \frac{\omega_R}{2\beta} \approx
				\frac{\omega_0}{\Delta\omega}$ (the latter is for lightly
				damped systems; $\Delta\omega$ is the distance between
				half-energy points --- $D_res/\sqrt{2}$ --- on the amplitude
				resonance curve) \\
		\end{mydescription}
	
	\subsubsection*{Underdamped oscillator}
		\begin{mydescription}
			\item[pseudo-frequency of oscillation:]
				$\omega_1^2 \equiv \omega_0^2 - \beta^2$  \\
			\item[solution (form 1):]
				$x(t) = e^{-\beta t}\left[ A_1e^{i\omega_1t} +
				A_2e^{-i\omega_1t} \right]$  \\
			\item[solution (form 2):]
				$x(t) = Ae^{-\beta t}\cos(\omega_1t-\delta)$  \\
			\item[phase plot:]
				Use the var. subst. $u=\omega_1 x$, $w=\beta x+\dot{x}$ and
				plot $w$ on the $y$-axis vs. $u$ on the $x$-axis \\
			\item[response to $\delta$ force:]
				$x(t) = \frac{b}{\omega_1}e^{-\beta
				(t-t_0)}\sin\omega_1(t-t_0)$  \\
			\item[green's fn:]
				$G(t,t') \equiv \frac{1}{m\omega_1}e^{-\beta
				(t-t')}\sin\omega_1(t-t')$, $t\geq t'$; $0$ otherwise  \\
		\end{mydescription}
	
	\subsubsection*{Critically damped oscillator}
		\begin{mydescription}
			\item[qualitative behavior:]
				System approaches equilibrium (natural solution dies out)
				faster than the others.\\
			\item[solution:]
				$x(t) = (A+Bt)e^{-\beta t}$  \\
		\end{mydescription}
	
	\subsubsection*{Overdamped oscillator}
		\begin{mydescription}
			\item[pseudo-frequency of (non-)oscillation:]
				$\omega_2^2 \equiv \beta^2 - \omega_0^2$  \\
			\item[solution:]
				$x(t) = e^{-\beta t}\left[ A_1e^{\omega_2t} + A_2e^{-\omega_2t}
				\right]$  \\
			\item[phase plot:]
				Asymptotic behavior tends towards $\dot{x}=-(\beta-\omega_2)x$
				unless $A_1=0$, then it goes to $\dot{x}=-(\beta+\omega_2)x$
		\end{mydescription}
	
	%\subsubsection*{Dirac Delta-Driven Underdamped Oscillator}
	%	\item[response to $\delta$ force:]
	%		$x(t) = \frac{b}{\omega_1}e^{-\beta (t-t_0)}\sin\omega_1(t-t_0)$  \\
	%	\item[green's fn:]
	%		$G(t,t') \equiv \frac{1}{m\omega_1}e^{-\beta
	%		(t-t')}\sin\omega_1(t-t')$, $t\geq t'$; $0$ otherwise  \\
	
	\subsubsection*{Series RLC circuit}
		\begin{mydescription}
			\item[voltage across inductor:]
				$V_L = L \frac{dI}{dt} = L\ddot{q}$  \\
			\item[voltage across resistor:]
				$V_R = IR = R\frac{dq}{dt} = R\dot{q}$  \\
			\item[voltage across capacitor:]
				$V_C = \frac{q}{C}$  \\
			\item[diffeq of RLC circuit with driving power source:]
				$L\ddot{q} + R\dot{q} + q/C = V(t)$  \\
		\end{mydescription}

	\subsection*{Electrical--mechanical equivalents}
		\begin{tabular}{l l l l}
			& \textbf{Mechanical} & & \textbf{Electrical} \\[3pt]
			\\[3pt]
			$x$       & Displacement            & $q$           & Charge \\
			[3pt]
			$\dot{x}$ & Velocity                & $\dot{q}=I$   & Current \\
			[3pt]
			$m$		  & Mass                    & $L$           & Inductance \\
			[3pt]
			$b$		  & Damping resistance 		& $R$			& Resistance \\
			[3pt]
			$l/k$	  & Mechanical compliance 	& $C$			& Capacitance \\
			[3pt]
			$F$		  & Ampl of impressed force & $\varepsilon$ &
												Ampl of impressed emf
		\end{tabular}
\end{multicols}
}\end{document}
