\documentclass[10pt,letterpaper,twoside]{article}
\usepackage[latin1]{inputenc}
\usepackage[pdftex]{graphicx}     % Add some packages for figures. Read epslatex.pdf on ctan.tug.org
\usepackage{geometry}
\usepackage{fancyhdr}
\usepackage{amsmath}
\usepackage{amsfonts}
\usepackage{bm}
\usepackage{outlines}
\usepackage{calligra}
%\usepackage{pxfonts}
%\usepackage[T1]{fontenc}{pxfonts}
%\usepackage{newpxtext,newpxmath}
\usepackage{mathrsfs}
\usepackage{units}
\usepackage{amssymb}
\usepackage{titlesec}
\usepackage[section]{placeins}
\usepackage{color}
\usepackage[labelfont={footnotesize,bf},font=footnotesize]{caption}
\usepackage[activate={true,nocompatibility},final,tracking=true,kerning=true,spacing=true,factor=1200,stretch=20,shrink=20]{microtype}
\usepackage[colorlinks=true,
			linkcolor=webgreen, %defined below
			filecolor=webbrown, %defined below
			citecolor=webgreen, %defined below
			%------------- Doc Info ---------------------------------
			pdftitle={Pre-algebra problems},
			pdfauthor={J. L. Lanfranchi},
			pdfsubject={},
			pdfkeywords={},
			%------------ Doc View ----------------------------------
			bookmarksopen=true,
			pdfpagemode=UseOutlines]{hyperref}
\definecolor{mygreen}{rgb}{0,0.6,0}
\definecolor{mygray}{rgb}{0.5,0.5,0.5}
\definecolor{mymauve}{rgb}{0.58,0,0.82}
\definecolor{webgreen}{rgb}{0.0,0,0.8}
\definecolor{webgreen}{rgb}{0.0,0,0.8}
\definecolor{webbrown}{rgb}{0,0,0.8}
\usepackage{calligra}
\DeclareMathAlphabet{\mathcalligra}{T1}{calligra}{m}{n}
\DeclareFontShape{T1}{calligra}{m}{n}{<->s*[2.2]callig15}{}
\newcommand{\scripty}[1]{\ensuremath{\mathcalligra{#1}}}

\setlength{\floatsep}{0.01in}
\setlength{\textfloatsep}{0.01in}
\setlength{\topmargin}{0.01in}
\setlength{\topskip}{0.01in}
\setlength{\textheight}{0.1in}
\setlength{\intextsep}{3pt}

\newcommand{\sectionlinetwo}[2]{%
  \nointerlineskip \vspace{.5\baselineskip}\hspace{\fill}
  {\resizebox{0.5\linewidth}{1.2ex}
    {\pgfornament[color = #1]{#2}
    }}%
    \hspace{\fill}
    \par\nointerlineskip \vspace{.5\baselineskip}
  }

%\author{J. L. Lanfranchi}
\title{Pre-algebra problems}
\geometry{top=0.70in,left=0.70in,right=0.70in,bottom=0.70in}
\begin{document} 
\twocolumn
\maketitle
\begin{enumerate}
    \item Solve for $x$ in the following: $$ 5\left(\frac{x^2}{3} - \frac{3}{4} \right) = \frac{35}{12} $$
    \item Peter has more than 150 books, but less than 200 books. Of these, 20\% are novels, and 1/7 are collections of poems. How many books does Peter have?
    \item Simplify the following expression
            $$\frac{1}{\frac{1}{x} + \frac{1}{y}}$$
        such that you end up with a single fraction where neither the numerator nor the
        denominator has a fraction in it.
    \item (Note that the following passage is basically copied from
        \url{https://en.wikipedia.org/wiki/Thin\_lens} but I've chosen a different sign
        convention, so equations won't look {\it exactly} the same as there.)

        The focal length of a lens $f$ is useful for determining how a lens will
        form images (how ``large'' they appear, and how far away an object must be for
        it to be in focus based on how far away you are from the lens).

        The focal length of a lens is determined by the lens's shape (geometry) and the
        material of which it is made (described by the material's index of refraction,
        $n$). The equation for this in ``geometric optics'' (things get more
        complicated, but save that for college) is given by the {\bf lensmaker's
        equation}:
            $$\frac{1}{f} = (n - 1)\left[ \frac{1}{R_1} + \frac{1}{R_2} - \frac{(n - 1)d}{nR_1R_2} \right]$$
        $f$ is the focal length, $n$ is the index of refraction of the lens's material, $R_1$ is the radius
        of curvature of what we'll call the ``front'' surface of the lens, $R_2$ is the radius of curvature of the
        ``back'' surface of the lens, and $d$ is the thickness of the lens.

        In the case that the thickness of the lens $d$ is ``very small'' compared to
        the radii of curvature, then $d/(R_1R_2)$ is going to be very small compared to
        either $1/R_1$ or $1/R_2$, and the
        above equation looks like
            $$\frac{1}{f} = (n - 1)\left[ \frac{1}{R_1} + \frac{1}{R_2} - {\rm relatively\;tiny\;number} \right]$$
        and so you can get relatively accurate results by dropping the last term.
        This gives you what's called the {\bf thin lens equation}:
            $$\frac{1}{f} \approx (n - 1)\left[ \frac{1}{R_1} + \frac{1}{R_2} \right]$$
        where $\approx$ means ``approximately equal to,'' but for simplicity you can just
        use the equals sign ($=$) from here on out.
            $$\frac{1}{f} = (n - 1)\left[ \frac{1}{R_1} + \frac{1}{R_2} \right]$$

        Now you have a simple equation you can work with. You'll note that the
        equation is written in terms of $\frac{1}{f}$ (called the ``reciprocal of $f$'')
        rather than $f$ probably because it's easier to see how to isolate $R_1$ and
        $R_2$ in this form than if you were to solve for $f$ (and reduce/simplify the
        resulting equation).

        Sometimes writing an equation in terms of the reciprocal (or some other
        expression) of the variable you ``care about'' makes it easier to work with than
        fully solving for the variable. When to leave an equation in one form not
        fully-solved for a variable depends on the situation, and you will get a better
        sense of when that's appropriate as you work more with equations.

        Anyway, there's still value in solving for $f$; do so, and also solve in terms
        of each of the following. Try to reduce the expressions as much as possible,
        e.g. using how you simplifieid the expression in Problem 3.
        
        \begin{enumerate}
          \item $f$
          \item $1/R_1$
          \item $R_1$
          \item $1/R_2$
          \item $R_2$
          \item $n$
        \end{enumerate}

        {\it Hint:} In solving for the reciprocal of $R_1$ (and likewise for the
        reciprocal of $R_2$), it may help to substitute a new letter for the reciprocal
        and work with that, just like you did for $x^2$ in Problem 1 (e.g., define $a_1
        \equiv 1/R_1$, solve for $a_1$, then substitute $1/R_1$ back in when you're
        done).

        Some following questions will involve working more with the thin lens equation, so
        keep your solutions around for those questions.

\end{enumerate}

\end{document}